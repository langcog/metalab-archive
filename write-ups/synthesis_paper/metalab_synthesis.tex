\documentclass[english,floatsintext,man]{apa6}

\usepackage{amssymb,amsmath}
\usepackage{ifxetex,ifluatex}
\usepackage{fixltx2e} % provides \textsubscript
\ifnum 0\ifxetex 1\fi\ifluatex 1\fi=0 % if pdftex
  \usepackage[T1]{fontenc}
  \usepackage[utf8]{inputenc}
\else % if luatex or xelatex
  \ifxetex
    \usepackage{mathspec}
    \usepackage{xltxtra,xunicode}
  \else
    \usepackage{fontspec}
  \fi
  \defaultfontfeatures{Mapping=tex-text,Scale=MatchLowercase}
  \newcommand{\euro}{€}
\fi
% use upquote if available, for straight quotes in verbatim environments
\IfFileExists{upquote.sty}{\usepackage{upquote}}{}
% use microtype if available
\IfFileExists{microtype.sty}{\usepackage{microtype}}{}

% Table formatting
\usepackage{longtable,booktabs}
\usepackage[counterclockwise]{rotating}   % Landscape page setup for large tables
\usepackage{multirow}		% Table styling
\usepackage{tabularx}		% Control Column width
\usepackage[flushleft]{threeparttable}	% Allows for three part tables with a specified notes section
\usepackage{threeparttablex}            % Lets threeparttable work with longtable
\usepackage{longtable}              % Allows tables to break across pages

  \usepackage{graphicx}
  \makeatletter
  \def\maxwidth{\ifdim\Gin@nat@width>\linewidth\linewidth\else\Gin@nat@width\fi}
  \def\maxheight{\ifdim\Gin@nat@height>\textheight\textheight\else\Gin@nat@height\fi}
  \makeatother
  % Scale images if necessary, so that they will not overflow the page
  % margins by default, and it is still possible to overwrite the defaults
  % using explicit options in \includegraphics[width, height, ...]{}
  \setkeys{Gin}{width=\maxwidth,height=\maxheight,keepaspectratio}
\ifxetex
  \usepackage[setpagesize=false, % page size defined by xetex
              unicode=false, % unicode breaks when used with xetex
              xetex]{hyperref}
\else
  \usepackage[unicode=true]{hyperref}
\fi
\hypersetup{breaklinks=true,
            pdfauthor={},
            pdftitle={A Quantitative Synthesis of Early Language Acquisition Using Meta-Analysis},
            colorlinks=true,
            citecolor=blue,
            urlcolor=blue,
            linkcolor=black,
            pdfborder={0 0 0}}
\urlstyle{same}  % don't use monospace font for urls

\setlength{\parindent}{0pt}
%\setlength{\parskip}{0pt plus 0pt minus 0pt}

\setlength{\emergencystretch}{3em}  % prevent overfull lines

\setcounter{secnumdepth}{0}
\ifxetex
  \usepackage{polyglossia}
  \setmainlanguage{}
\else
  \usepackage[english]{babel}
\fi

% Manuscript styling
\captionsetup{font=singlespacing,justification=justified}
\usepackage{csquotes}



\usepackage{tikz} % Variable definition to generate author note

% fix for \tightlist problem in pandoc 1.14
\providecommand{\tightlist}{%
  \setlength{\itemsep}{0pt}\setlength{\parskip}{0pt}}

% Essential manuscript parts
  \title{A Quantitative Synthesis of Early Language Acquisition Using
Meta-Analysis}

  \shorttitle{A Quantitative Synthesis}


  \author{
          Molly Lewis\textsuperscript{1},
          Mika Braginsky\textsuperscript{1},
          Sho Tsuji\textsuperscript{2},
          Christina Bergmann\textsuperscript{2},
          Page Piccinini\textsuperscript{2},
          Alejandrina Cristia\textsuperscript{2},
          Michael C. Frank\textsuperscript{1}  }

  \def\affdep{{"", "", "", "", "", "", ""}}%
  \def\affcity{{"", "", "", "", "", "", ""}}%

  \affiliation{
    \vspace{0.5cm}
          \textsuperscript{1} Department Psychology, Stanford University\\
          \textsuperscript{2} Laboratoire de Sciences Cognitives et Psycholinguistique, ENS  }


%   \def\affinst{{"init", "Department Psychology, Stanford University", "Laboratoire de Sciences Cognitives et Psycholinguistique, ENS"}}%
%   \def\affstate{{"init", "", ""}}%
%   \def\affcntry{{"init", "", ""}}%

  \note{
    \vspace{1cm}
    Author note

    \raggedright
    \setlength{\parindent}{0.4in}

    \newcounter{author}

%     %     %       %       \setcounter{author}{0}
%         %           \addtocounter{author}{1}
%         %         \expandafter\edef\csname authorid\endcsname{\theauthor}
%         Molly Lewis, \pgfmathparse{\affdep[\authorid]} \pgfmathresult, \pgfmathparse{\affinst[\authorid]} \pgfmathresult, \pgfmathparse{\affcity[\authorid]} \pgfmathresult, \pgfmathparse{\affstate[\authorid]} \pgfmathresult, \pgfmathparse{\affcntry[\authorid]} \pgfmathresult
%       %     ;
%     %       %       \setcounter{author}{0}
%         %           \addtocounter{author}{1}
%         %         \expandafter\edef\csname authorid\endcsname{\theauthor}
%         Mika Braginsky, \pgfmathparse{\affdep[\authorid]} \pgfmathresult, \pgfmathparse{\affinst[\authorid]} \pgfmathresult, \pgfmathparse{\affcity[\authorid]} \pgfmathresult, \pgfmathparse{\affstate[\authorid]} \pgfmathresult, \pgfmathparse{\affcntry[\authorid]} \pgfmathresult
%       %     ;
%     %       %       \setcounter{author}{0}
%         %           \addtocounter{author}{2}
%         %         \expandafter\edef\csname authorid\endcsname{\theauthor}
%         Sho Tsuji, \pgfmathparse{\affdep[\authorid]} \pgfmathresult, \pgfmathparse{\affinst[\authorid]} \pgfmathresult, \pgfmathparse{\affcity[\authorid]} \pgfmathresult, \pgfmathparse{\affstate[\authorid]} \pgfmathresult, \pgfmathparse{\affcntry[\authorid]} \pgfmathresult
%       %     ;
%     %       %       \setcounter{author}{0}
%         %           \addtocounter{author}{2}
%         %         \expandafter\edef\csname authorid\endcsname{\theauthor}
%         Christina Bergmann, \pgfmathparse{\affdep[\authorid]} \pgfmathresult, \pgfmathparse{\affinst[\authorid]} \pgfmathresult, \pgfmathparse{\affcity[\authorid]} \pgfmathresult, \pgfmathparse{\affstate[\authorid]} \pgfmathresult, \pgfmathparse{\affcntry[\authorid]} \pgfmathresult
%       %     ;
%     %       %       \setcounter{author}{0}
%         %           \addtocounter{author}{2}
%         %         \expandafter\edef\csname authorid\endcsname{\theauthor}
%         Page Piccinini, \pgfmathparse{\affdep[\authorid]} \pgfmathresult, \pgfmathparse{\affinst[\authorid]} \pgfmathresult, \pgfmathparse{\affcity[\authorid]} \pgfmathresult, \pgfmathparse{\affstate[\authorid]} \pgfmathresult, \pgfmathparse{\affcntry[\authorid]} \pgfmathresult
%       %     ;
%     %       %       \setcounter{author}{0}
%         %           \addtocounter{author}{2}
%         %         \expandafter\edef\csname authorid\endcsname{\theauthor}
%         Alejandrina Cristia, \pgfmathparse{\affdep[\authorid]} \pgfmathresult, \pgfmathparse{\affinst[\authorid]} \pgfmathresult, \pgfmathparse{\affcity[\authorid]} \pgfmathresult, \pgfmathparse{\affstate[\authorid]} \pgfmathresult, \pgfmathparse{\affcntry[\authorid]} \pgfmathresult
%       %     ;
%     %       %       \setcounter{author}{0}
%         %           \addtocounter{author}{1}
%         %         \expandafter\edef\csname authorid\endcsname{\theauthor}
%         Michael C. Frank, \pgfmathparse{\affdep[\authorid]} \pgfmathresult, \pgfmathparse{\affinst[\authorid]} \pgfmathresult, \pgfmathparse{\affcity[\authorid]} \pgfmathresult, \pgfmathparse{\affstate[\authorid]} \pgfmathresult, \pgfmathparse{\affcntry[\authorid]} \pgfmathresult
%       %     .
%     
    Correspondence concerning this article should be addressed to Molly
    Lewis, Psychology Department, Stanford University. 450 Serra Mall,
    Stanford, CA 94305. E-mail:
    \href{mailto:mll@stanford.edu}{\nolinkurl{mll@stanford.edu}}.

                                                                                    }

  \abstract{replicability, etc.}
  \keywords{replicability, reproducibility, meta-analysis, developmental psychology,
language acquisition \\

    \indent Word count: XXXX
  }

  \usepackage{setspace}
  \AtBeginEnvironment{tabular}{\singlespacing}
  \usepackage{pbox}

\begin{document}

\maketitle



\section{Introduction}\label{introduction}

To learn to speak a language, a child must acquire a wide range
knowledge and skills: the sounds of the language, the word forms, and
how words map to meanings, to name only a few. How does this process
unfold? Our goal as psychologists is to build a theory that can explain
and predict this process in a way that is both precise but also highly
generalizable. The challenge we face, however, is that we must build
this theory on the basis of very limited data, and, as a consequence, we
must rely on error-prone inductive reasoning strategies to build our
theories. In this paper, we consider the domain of language acquisition
and demonstrate how meta-analytic methods can support the inductive
theory building process.

Meta-analysis is a quantitative method for aggregating across
experimental findings. The fundamental unit of meta-analysis is the
\emph{effect size}: a scale-free, quantitative measure of
\enquote{success} in a phenomenon. Importantly, an effect size provides
an estimate of the \emph{size} of an effect, as well as a measure of
uncertainty around this point estimate. With such a quantitative measure
of success, we can apply the same reasoning we use to aggregate noisy
measurements over participants in a single study: By assuming each
\emph{study}, rather than participant, is sampled from a population, we
can appeal to the classical statistical framework to combine estimates
of the effect size for a given phenomenon.

Meta-analytic methods support theory building in several ways. First,
they provide a way of evaluating which effects in a literature are
\enquote{real,} and thus should constrain the theory. This is
particulary important in light of recent high-profile evidence in the
field that an effect observed in one study may not replicate in another
(Collaboration \& others, 2012, 2015; ``replication crisis'', Ioannidis,
2005). Failed replications are difficult to interpret, however, because
they may result from a wide variety of causes, including an in initial
false positive, a subsequent false negative, or differences between
initial and replication studies, and making causal attributions in a
situation with two conflicting studies is often difficult (Gilbert et
al., 2016; Anderson et al., 2016). Meta-analysis can allow researchers
to address this set of issues in a principled way by aggregating
evidence across studies and assuming that there is some variability in
true effect size from study to study. In this way, meta-analytic methods
can provide more veridical description of the empirical landscape, which
in turn leads to better theory-building.

Second, meta-analysis supports theory building by providing higher
fidelity descriptions of phenomenona. Rather than simply concluding that
an effect exists, effect sizes allow us to ask finer grain questions:
How much variability is there in the effect? How does the effect change
over development? To what extent does a moderator of theoretetic
influence the effect? This type of continuous analysis supports building
quantitative models, and specifying theories that are more precise and
constraining.

Furthermore, effect sizes provide a common language of comparing
\emph{across} phenomena. This common language allows us to meaningfully
consider the relationship between different phenomena in the language
acquisition domain (\enquote{meta-meta-analysis}). Through
cross-phenomena comparions, we can understand not only the trajectory of
a particular phenomenon, like word learning for example, but also how
this phenomenon might depend on other skills, such as sound learning,
gaze following, and many others. With this more complete picture of the
interaction of different linguistic competencies, we can begin to build
a more synthetic theory of language acquisition.

Finally, in addition to these theoretical motivations, there are
practical reasons for conducting a quantitative synthesis. When planning
an experiment, an estimate of the size of an effect on the basis of
prior literature can inform the sample size needed to achieve a desired
level of power. Meta-analytic estimates of effect sizes can also aid in
design choices: If a certain paradigm tends to have overall larger
effect sizes than another, the strategic researcher might select this
paradigm in order to maximize the power of a study.

While meta-analytic methods are likely helpful for many psychological
literatures, we believe language acquisition is a particularly
informative application for this tool. One reason is that language
acquistion may be uniquely vulnerable to false findings because running
children is expensive, and thus sample sizes are small and studies are
underpowered (Ioannidis, 2005). In addition, the difficulty in running
participants means that replications are relatively rare in the field.
Finally, there has been attention to developmental psychology research
practices more broadly, suggesting evidence of experimenter bias
(Peterson, 2016).

We take as our ultimate goal a single overarching theory of language
acquisition that can explain and predict all the relevant pheonemona.
Toward this end, we developed a dataset of effect sizes in the language
acquistion literature across 12 different phenomenona
(\href{http://metalab.stanford.edu}{Metalab;
http://metalab.stanford.edu/}). We demonstrate how meta-analysis
supports building this theory in two ways. We first use meta-analytic
techniques to evaluate the evidential value of the empirical landscape
in language acquistion research. We find broadly that this literature
has strong evidential value, and thus that the effects report in the
literature should constrain our theory of language acquisition. We then
turn to synthesizing these findings across phenenomena and offer a
preliminary theoretical synthesis of the field.

\section{Method}\label{method}

We analyzed 12 different phenomenena in language acquisition. These
phenomena were selected opportunistically, based on their availability
in the literature and feasibility of conduciting meta-analysis for a
particular phenomenon. The phenomenena cover development at many
different levels of the language hierarchy, from the acquistion of
prosody and phonemic contrasts, to gaze following in linguistic
interaction. This wide range of phenomena allowed us to compare the
course of development across different domains, as well as explore
questions about the interactive nature of language acquisition (Table
1).

To obtain estimates of effect size, we coded papers reporting
experimental data. Within each paper, we calculated a separate effect
size estimate for each experiment and age group (\enquote{condition}).
In total, our sample includes estimates from 269 papers, 981 different
conditions and 12,029 participants. The process for selecting papers
from the literature differed by domain, with some individual
meta-analyses using more systematic approaches than others (see SI).
\renewcommand{\arraystretch}{1.5}

\begin{table}[h!]
        \footnotesize
        \begin{tabular}{lp{4cm} p{5cm}r}
            \toprule
            \textbf{Level} & \textbf{Phenomenon}                                                               & \textbf{Description}                                                                                 & \textbf{N papers (conditions)}                                                                                                                                               \\
                        \midrule

            Prosody        & IDS  preference  \newline  {\scriptsize (Dunst, Gorman, \& Hamby, 2012)}          & {\scriptsize  Looking times as a function of whether infant-directed vs. adult-directed speech is presented as stimulation.}      & 16 (50)     \\
            Sounds         & Phonotactic learning  \newline {\scriptsize (Cristia, in prep.)}                   & {\scriptsize Infants' ability to learn phonotactic generalizations from a short exposure.  }                  & 15 (47)                               \\
            ~              & Vowel discrimination (native) \newline {\scriptsize (Tsuji \& Cristia, 2014)}     & {\scriptsize Discrimination of native-language vowels, including results from a variety of methods.  }         & 40 (167)             \\ 
            ~              & Vowel discrimination (non-native) \newline {\scriptsize (Tsuji \& Cristia, 2014)} & {\scriptsize Discrimination of non-native vowels, including results from a variety of methods.  }     & 21 (72)     \\
               & Statistical sound learning  \newline {\scriptsize (Cristia, in prep.)}             & {\scriptsize Infants' ability to learn sound categories from their acoustic distribution.   }  & 11 (40) \\ 
            & Word segmentation \newline {\scriptsize  (Bergmann \& Cristia, 2015) }            & {\scriptsize Recognition of familiarized words from running, natural speech using behavioral methods.  }                     & 68 (296)                                     \\
            Words     &   Mutual exclusivity \newline {\scriptsize (Lewis \& Frank, in prep.)} &{\scriptsize  Mapping of novel words reflecting children's inference that novel words tend to refer to novel objects.}
            & 20 (60)             \\
            ~ &   Sound Symbolism \newline {\scriptsize (Lammertink et al., in prep.)} &{\scriptsize  Non-arbitrary relationship between form and meaning ("bouba-kiki effect").}
            & 10 (42)             \\
            ~              & Concept-label advantage   \newline {\scriptsize (Lewis \& Long, unpublished)}     & {\scriptsize Infants' categorization judgments in the presence and absence of labels.    } & 16 (100) \\
            ~              & Online word recognition \newline {\scriptsize (Frank, Lewis, \& MacDonald, 2016)} & {\scriptsize Online word recognition of familiar words using two-alternative forced choice preferential looking.   }              & 12 (32)                         \\
            Communication  & Gaze following  \newline {\scriptsize  (Frank, Lewis, \& MacDonald, 2016)}        & {\scriptsize Gaze following using standard multi-alternative forced-choice paradigms.   }                       & 15 (45)                                           \\
            ~              & Pointing and vocabulary  \newline {\scriptsize (Colonnesi et al., 2010)}          & {\scriptsize Longitudinal correlations between declarative pointing and later vocabulary.  }               & 25 (30)                         \\ 
            \bottomrule
        \end{tabular}
        \caption{Overview of meta-analyses in dataset.}
    \end{table}

\section{Replicability of the field}\label{replicability-of-the-field}

To assess the replicability of language acquisition phenomena, we
conducted several diagnostic analyses: Meta-analytic estimates of effect
size, fail-safe-N (Orwin, 1983), funnel plots, and p-curve (U.
Simonsohn, Nelson, \& Simmons, 2014; Simonsohn, Nelson, \& Simmons,
2014; Simonsohn, Simmons, \& Nelson, 2015). These analytical approaches
each have limitations, but taken together, they provide converging
evidence about the replicability of a literature. Overall, we find
little evidence of bias in our meta-analyses, suggesting that the
language acqusition literature likely describes real psychological
phenomenona and should therefore provide the basis for theoretical
development.

\subsection{Meta-Analytic Effect Size}\label{meta-analytic-effect-size}

To estimate the overall effect size of a literature, effect sizes are
pooled across papers to obtain a single meta-analytic estimate. This
meta-analytic effect-size can be thought of as the ``best estimate" of
the effect size for a phenomenon given all the available data in the
literature.

Table 2, column 2 presents meta-analytic effect size estimates for each
of our phenomenona. We find evidence for a non-zero effect size in 11
out of 12 of our phenomena, suggesting these literature provide
evidential value. In the case of phonotatic learning, however, we find
that the meta-analytic effect size estimate does not differ from zero,
suggest that this literature does not describe a robust effect.
{[}Remove it from analyses below?{]}.

While the meaure of effect size is itself quantitative, meta-analytic
estimates of effect size provide only categorical information about the
evidential value of a literature: the effect is real, or not. But, a
more powerful method of assessing evidential value would tell us the
\emph{degree} to which a literature has evidential value, and thus the
degree to which it should constrain our theory building. In the
following three analyses---fail-safe-N, funnel plots, and p-curves---we
describe through analyses that quantify the evidential value of these
literatures.

\subsection{Fail-safe-N}\label{fail-safe-n}

One approach for quantifying the reliability of a literature is to ask,
How many missing studies with null effects would have to exist in the
\enquote{file drawer} in order for the overall effect size to be zero?
This is called the \enquote{fail-safe} number of studies (Orwin, 1983).
To answer this question, we estimated the overall effect size for each
phenomenenon (Table 2, column 2), and then used this to estimate the
fail-safe-N (Table 2, column 3).

Because of the large number of positive studies in many of the MAs we
assessed, this analysis suggests a very large number of studies would
have to be \enquote{missing} in each literature (\(M\) = 3634) in order
for the overall effect sizes to be 0. Thus, while it is possible that
some reporting bias is present in the literature, the large fail-safe-N
suggests that the literature nonetheless likely describes a real effect.

One limitation of this analysis, however, is that it assumes that all
reported effect sizes are obtained in the absence of analytical
flexibility: If experimenters are exercising analytical flexibility
through practices like p-hacking, then the number and magnitude of
observed true effects in the literature may be inflated. In the next
analysis, we examine this possibility through funnel plots.

\subsection{Funnel Plots}\label{funnel-plots}

Funnel plots provide a visual method for evaluating whether variability
in effect sizes is due only to differences in sample size. A funnel plot
shows effect sizes versus a metric of sample size, standard error. If
there is no bias in a literature, we should expect studies to be
randomly sampled around the mean, with more variability for less precise
studies.

Figure 1 presents funnel plots for each of our 12 meta-analyses. These
plots show evidence of asymmetry (bias) for several of our phenomenon
(Table 2, column 4). However, an important limitation of this method is
that it is difficult to determine the source of this bias. One
possibility is that this bias reflects true heterogenity in phenomena
(e.g.~different ages). P-curve analyses provide one method for
addressing this issue, which we turn to next.

\begin{figure}[htbp]
\centering
\includegraphics{metalab_synthesis_files/figure-latex/unnamed-chunk-2-1.pdf}
\caption{Funnel plots for each meta-analysis. Each effect size estimate
is represented by a point, and the mean effect size is shown as a red
dashed line. The funnel corresponds to a 95\% (narrow) and 99\% (wide)
CI around this mean. In the absense of true heterogenity in effect sizes
(no moderators) and bias, we should expect all points to fall inside the
funnel.}
\end{figure}

\subsection{P-curves}\label{p-curves}

\begin{figure}[htbp]
\centering
\includegraphics{metalab_synthesis_files/figure-latex/p_curve_plots-1.pdf}
\caption{P-curve for each meta-analysis (Simonsohn, Nelson, \& Simmons,
2014). In the absense of p-hacking, we should expect the observed
p-curve (blue) to be right-skewed (more small values). The red dashed
line shows the expected distribution of p-values when the effect is
non-existent (the null is true). The green dashed line shows the
expected distribution if the effect is real, but studies only have 33\%
power.}
\end{figure}

A p-curve is the distribution of p-values for the statistical test of
the main hypothesis across a literature (U. Simonsohn et al., 2014;
Simonsohn et al., 2014, 2015). Critically, if there is a robust effect
in the literature, the shape of the p-curve should reflect this. In
particular, we should expect the p-curve to be right skewed with more
small values (e.g., .01) than large values (e.g., .04). An important
property of this analysis is that we should expect this skew independent
of any true heterogenity in the data, such as age. Evidence that the
curve is in fact right-skewed would suggest that the literature is not
biased, and that it provides evidential value for theory building.

\begin{table}[t]
\footnotesize
\begin{tabular}{lrrrrr}
\toprule
\textbf{Phenomenon}& \textbf{\textit{d}} & \textbf{fail-safe-N} & \textbf{funnel skew} & \textbf{p-curve skew} & \textbf{power}\\
\midrule
IDS preference & 0.71 [0.53, 0.89] & 3762 & 1.88 (0.06) &  & \\
Phonotactic learning & 0.04 [-0.09, 0.16] & 45 & -1.08 (0.28) & -1.52 (0.06) & 0.14\\
Vowel discrimination (native) & 0.6 [0.5, 0.71] & 9536 & 8.98 (0) & -5.42 (0) & 0.67\\
Vowel discrimination (non-native) & 0.66 [0.42, 0.9] & 3391 & 4.13 (0) & -3.24 (0) & 0.78\\
Statistical sound learning & -0.14 [-0.27, -0.02] & Inf & -1.87 (0.06) &  & \\
Word segmentation & 0.2 [0.15, 0.25] & 5645 & 1.54 (0.12) & -9.67 (0) & 0.56\\
Mutual exclusivity & 1.01 [0.68, 1.33] & 6443 & 6.25 (0) &  & \\
Sound symbolism & 0.15 [0.04, 0.26] & 538 & -1.32 (0.19) & -2.16 (0.02) & 0.96\\
Concept-label advantage & 0.4 [0.29, 0.51] & 3928 & 0.31 (0.76) & -6.15 (0) & 0.69\\
Online word recognition & 1.89 [0.81, 2.96] & 2843 & 2.92 (0) &  & \\
Gaze following & 0.84 [0.26, 1.42] & 2641 & -1.69 (0.09) &  & \\
Pointing and vocabulary & 0.41 [0.32, 0.49] & 1202 & 0.59 (0.55) &  & \\
\bottomrule
\end{tabular}
\caption{Summary of replicability analyses. \textit{d} = Effect size (Cohen's {\it d}) estimated from a random-effect model; fail-safe-N = number of missing studies that would have to exist in order for the overall effect size to be {\it d} = 0; funnel skew = test of asymmetry in funnel plot using the random-effect Egger's test (Stern \& Eggers, 2005); p-curve skew = test of the right skew of the p-curve using the Stouffer method (Simonsohn, Simmons, \& Nelson, 2015); power = power to reject the null hypothesis at the 5\% significance level based on the p-curve (Simonsohn, Nelson, \& Simmons, 2014);  Brackets give 95\% confidence intervals, and parentheses show p-values.}
\end{table}

Figure 2 shows p-curves for 7 of our 12
meta-analyses.\footnote{We did not conduct p-curves on all meta-analyses because previously published meta-analyses did not include test statistics and the key test statistics in some others were inappropriate for p-curve. }
With the exception of phonotactic learning, all p-curves show evidence
of right skew. This is confirmed by formal analyses (Table 2, column 5).

P-curves also provide a method for calculcating the overall power of a
literature, based on the shape of the p-curve (U. Simonsohn et al.,
2014). Intuitively, when power is high and effect is real, we should be
more likely to observe an effect size \enquote{further} from the null.
This means that we will observe more small effect sizes. Thus, the
higher the power, the more right skewed the p-curve will be. Table 2
(column 6) presents estimates of power for each meta-analysis based on
p-curve. With the exception of phonotactic learning (\emph{power} =
.14), literatures appear to have acceptable power.

\section{Theoretical Synthesis}\label{theoretical-synthesis}

\begin{figure}[htbp]
\centering
\includegraphics{metalab_synthesis_files/figure-latex/unnamed-chunk-4-1.pdf}
\caption{Meta-meta plot}
\end{figure}

\singlespacing

\begin{itemize}
\item
  Ultimately, what we would like to have, is a theory of how children
  acquire language at all levels of the linguistic hierarchy.
\item
  There are a number of different theories you could have.
\item
  One possibilty is the \enquote{stages} hypothesis, where children
  learn language representations in a sequential fashion starting at the
  lowest levels of the hierarchy.
\item
  Another possibility though is a more continous, synegistic theory,
  where learning at all levels is happening simultaneously.
\item
  And there are a number of recent pieces of evidence that are
  consisntent with this. For example,

  \begin{itemize}
  \tightlist
  \item
    Infants learn phonetic contrasts when supported by word context
    (Feldman, et al., 2013).
  \item
    Infants learn word mappings when supported by prosody (Shukla,
    White, \& Aslin, 2011).
  \end{itemize}
\item
  But, the hypothesis space is really quite large, once you start
  thinking about it in terms of ES {[}plots of hypothesis space; e.g.,
  \url{https://speakerdeck.com/mllewis/metalab-1?slide=32}{]}

  \begin{itemize}
  \tightlist
  \item
    For example, you might you learn low levels stuff like prosody
    first, then sounds, then words, then communicaiton
  \item
    Another possibility is that you don't have to quite master a skill
    before learning gets off the ground
  \item
    Learning also doesn't have to monotonic in these tasks -- children
    in a task might get \enquote{worse,} for example because the skill
    is no longer relevant (like non-native vowels)
  \end{itemize}
\item
  Fig. 3 shows what this looks like based our dataset
\item
  Evidence for interactivity!
\end{itemize}

\doublespacing

\section{Discussion}\label{discussion}

Limitations

\paragraph{Author Contributions}\label{author-contributions}

\paragraph{Acknowledgments}\label{acknowledgments}

\newpage

\subsubsection{References}\label{references}

\setlength{\parindent}{-0.5in} \setlength{\leftskip}{0.5in}
\setlength{\parskip}{8pt}

\hypertarget{refs}{}
\hypertarget{ref-bergmann2015development}{}
Bergmann, C., \& Cristia, A. (2015). Development of infants'
segmentation of words from native speech: A meta-analytic approach.
\emph{Developmental Science}.

\hypertarget{ref-open2012open}{}
Collaboration, O. S., \& others. (2012). An open, large-scale,
collaborative effort to estimate the reproducibility of psychological
science. \emph{Perspectives on Psychological Science}, \emph{7}(6),
657--660.

\hypertarget{ref-open2015estimating}{}
Collaboration, O. S., \& others. (2015). Estimating the reproducibility
of psychological science. \emph{Science}, \emph{349}(6251), aac4716.

\hypertarget{ref-dunst2012preference}{}
Dunst, C., Gorman, E., \& Hamby, D. (2012). Preference for
infant-directed speech in preverbal young children. \emph{Center for
Early Literacy Learning}, \emph{5}(1).

\hypertarget{ref-frank2016performance}{}
Frank, M. C., Lewis, M. L., \& MacDonald, K. (in press). A performance
model for early word learning. In \emph{Proceedings of the 38th Annual
Conference of the Cognitive Science Society}. Retrieved from
\url{http://langcog.stanford.edu/papers_new/frank-2016-underrev.pdf}

\hypertarget{ref-ioannidis2005most}{}
Ioannidis, J. P. (2005). Why most published research findings are false.
\emph{PLoS Med}, \emph{2}(8), e124.

\hypertarget{ref-lfprep}{}
Lewis, M., \& Frank, M. C. (in prep). Multiple routes to disambiguation.

\hypertarget{ref-orwin1983fail}{}
Orwin, R. G. (1983). A fail-safe n for effect size in meta-analysis.
\emph{Journal of Educational Statistics}, 157--159.

\hypertarget{ref-Peterson:2016}{}
Peterson, D. (2016). The Baby Factory: Difficult Research Objects,
Disciplinary Standards, and the Production of Statistical Significance.
\emph{Socius: Sociological Research for a Dynamic World}, \emph{2}(0),
1--10.

\hypertarget{ref-simonsohn2014power}{}
Simonsohn, U., Nelson, L. D., \& Simmons, J. P. (2014). P-curve and
effect size correcting for publication bias using only significant
results. \emph{Perspectives on Psychological Science}, \emph{9}(6),
666--681.

\hypertarget{ref-simonsohn2014p}{}
Simonsohn, Nelson, L. D., \& Simmons, J. P. (2014). P-curve: A key to
the file-drawer. \emph{Journal of Experimental Psychology: General},
\emph{143}(2), 534.

\hypertarget{ref-simonsohn2015better}{}
Simonsohn, Simmons, J. P., \& Nelson, L. D. (2015). Better p-curves.
\emph{Simonsohn, Uri, Joseph P. Simmons, and Leif D. Nelson
(Forthcoming),``Better P-Curves,'' Journal of Experimental Psychology:
General}.

\hypertarget{ref-sterne2005regression}{}
Sterne, J. A., \& Egger, M. (2005). Regression methods to detect
publication and other bias in meta-analysis. \emph{Publication Bias in
Meta-Analysis: Prevention, Assessment, and Adjustments}, 99--110.

\hypertarget{ref-tsuji2014perceptual}{}
Tsuji, S., \& Cristia, A. (2014). Perceptual attunement in vowels: A
meta-analysis. \emph{Developmental Psychobiology}, \emph{56}(2),
179--191.



\end{document}
