\documentclass[american,floatsintext,man]{apa6}

\usepackage{amssymb,amsmath}
\usepackage{ifxetex,ifluatex}
\usepackage{fixltx2e} % provides \textsubscript
\ifnum 0\ifxetex 1\fi\ifluatex 1\fi=0 % if pdftex
  \usepackage[T1]{fontenc}
  \usepackage[utf8]{inputenc}
\else % if luatex or xelatex
  \ifxetex
    \usepackage{mathspec}
    \usepackage{xltxtra,xunicode}
  \else
    \usepackage{fontspec}
  \fi
  \defaultfontfeatures{Mapping=tex-text,Scale=MatchLowercase}
  \newcommand{\euro}{€}
\fi
% use upquote if available, for straight quotes in verbatim environments
\IfFileExists{upquote.sty}{\usepackage{upquote}}{}
% use microtype if available
\IfFileExists{microtype.sty}{\usepackage{microtype}}{}

% Table formatting
\usepackage{longtable,booktabs}
\usepackage[counterclockwise]{rotating}   % Landscape page setup for large tables
\usepackage{multirow}		% Table styling
\usepackage{tabularx}		% Control Column width
\usepackage[flushleft]{threeparttable}	% Allows for three part tables with a specified notes section
\usepackage{threeparttablex}            % Lets threeparttable work with longtable
\usepackage{longtable}              % Allows tables to break across pages

\ifxetex
  \usepackage[setpagesize=false, % page size defined by xetex
              unicode=false, % unicode breaks when used with xetex
              xetex]{hyperref}
\else
  \usepackage[unicode=true]{hyperref}
\fi
\hypersetup{breaklinks=true,
            pdfauthor={},
            pdftitle={A Quantitative Synthesis of Early Language Acquisition Using Meta-Analysis},
            colorlinks=true,
            citecolor=blue,
            urlcolor=blue,
            linkcolor=black,
            pdfborder={0 0 0}}
\urlstyle{same}  % don't use monospace font for urls

\setlength{\parindent}{0pt}
%\setlength{\parskip}{0pt plus 0pt minus 0pt}

\setlength{\emergencystretch}{3em}  % prevent overfull lines

\setcounter{secnumdepth}{0}
\ifxetex
  \usepackage{polyglossia}
  \setmainlanguage{}
\else
  \usepackage[american]{babel}
\fi

% Manuscript styling
\captionsetup{font=singlespacing,justification=justified}
\usepackage{csquotes}



\usepackage{tikz} % Variable definition to generate author note

% fix for \tightlist problem in pandoc 1.14
\providecommand{\tightlist}{%
  \setlength{\itemsep}{0pt}\setlength{\parskip}{0pt}}

% Essential manuscript parts
  \title{A Quantitative Synthesis of Early Language Acquisition Using
Meta-Analysis}

  \shorttitle{A Quantitative Synthesis}


  \author{
          Molly Lewis\textsuperscript{1},
          Mika Braginsky\textsuperscript{1},
          Sho Tsuji\textsuperscript{2},
          Christina Bergmann\textsuperscript{2},
          Page Piccinini\textsuperscript{2},
          Alejandrina Cristia\textsuperscript{2},
          Michael C. Frank\textsuperscript{1}  }

  \def\affdep{{"", "", "", "", "", "", ""}}%
  \def\affcity{{"", "", "", "", "", "", ""}}%

  \affiliation{
    \vspace{0.5cm}
          \textsuperscript{1} Department Psychology, Stanford University\\
          \textsuperscript{2} Laboratoire de Sciences Cognitives et Psycholinguistique, ENS  }


%   \def\affinst{{"init", "Department Psychology, Stanford University", "Laboratoire de Sciences Cognitives et Psycholinguistique, ENS"}}%
%   \def\affstate{{"init", "", ""}}%
%   \def\affcntry{{"init", "", ""}}%

  \note{
    \vspace{1cm}
    Author note

    \raggedright
    \setlength{\parindent}{0.4in}

    \newcounter{author}

%     %     %       %       \setcounter{author}{0}
%         %           \addtocounter{author}{1}
%         %         \expandafter\edef\csname authorid\endcsname{\theauthor}
%         Molly Lewis, \pgfmathparse{\affdep[\authorid]} \pgfmathresult, \pgfmathparse{\affinst[\authorid]} \pgfmathresult, \pgfmathparse{\affcity[\authorid]} \pgfmathresult, \pgfmathparse{\affstate[\authorid]} \pgfmathresult, \pgfmathparse{\affcntry[\authorid]} \pgfmathresult
%       %     ;
%     %       %       \setcounter{author}{0}
%         %           \addtocounter{author}{1}
%         %         \expandafter\edef\csname authorid\endcsname{\theauthor}
%         Mika Braginsky, \pgfmathparse{\affdep[\authorid]} \pgfmathresult, \pgfmathparse{\affinst[\authorid]} \pgfmathresult, \pgfmathparse{\affcity[\authorid]} \pgfmathresult, \pgfmathparse{\affstate[\authorid]} \pgfmathresult, \pgfmathparse{\affcntry[\authorid]} \pgfmathresult
%       %     ;
%     %       %       \setcounter{author}{0}
%         %           \addtocounter{author}{2}
%         %         \expandafter\edef\csname authorid\endcsname{\theauthor}
%         Sho Tsuji, \pgfmathparse{\affdep[\authorid]} \pgfmathresult, \pgfmathparse{\affinst[\authorid]} \pgfmathresult, \pgfmathparse{\affcity[\authorid]} \pgfmathresult, \pgfmathparse{\affstate[\authorid]} \pgfmathresult, \pgfmathparse{\affcntry[\authorid]} \pgfmathresult
%       %     ;
%     %       %       \setcounter{author}{0}
%         %           \addtocounter{author}{2}
%         %         \expandafter\edef\csname authorid\endcsname{\theauthor}
%         Christina Bergmann, \pgfmathparse{\affdep[\authorid]} \pgfmathresult, \pgfmathparse{\affinst[\authorid]} \pgfmathresult, \pgfmathparse{\affcity[\authorid]} \pgfmathresult, \pgfmathparse{\affstate[\authorid]} \pgfmathresult, \pgfmathparse{\affcntry[\authorid]} \pgfmathresult
%       %     ;
%     %       %       \setcounter{author}{0}
%         %           \addtocounter{author}{2}
%         %         \expandafter\edef\csname authorid\endcsname{\theauthor}
%         Page Piccinini, \pgfmathparse{\affdep[\authorid]} \pgfmathresult, \pgfmathparse{\affinst[\authorid]} \pgfmathresult, \pgfmathparse{\affcity[\authorid]} \pgfmathresult, \pgfmathparse{\affstate[\authorid]} \pgfmathresult, \pgfmathparse{\affcntry[\authorid]} \pgfmathresult
%       %     ;
%     %       %       \setcounter{author}{0}
%         %           \addtocounter{author}{2}
%         %         \expandafter\edef\csname authorid\endcsname{\theauthor}
%         Alejandrina Cristia, \pgfmathparse{\affdep[\authorid]} \pgfmathresult, \pgfmathparse{\affinst[\authorid]} \pgfmathresult, \pgfmathparse{\affcity[\authorid]} \pgfmathresult, \pgfmathparse{\affstate[\authorid]} \pgfmathresult, \pgfmathparse{\affcntry[\authorid]} \pgfmathresult
%       %     ;
%     %       %       \setcounter{author}{0}
%         %           \addtocounter{author}{1}
%         %         \expandafter\edef\csname authorid\endcsname{\theauthor}
%         Michael C. Frank, \pgfmathparse{\affdep[\authorid]} \pgfmathresult, \pgfmathparse{\affinst[\authorid]} \pgfmathresult, \pgfmathparse{\affcity[\authorid]} \pgfmathresult, \pgfmathparse{\affstate[\authorid]} \pgfmathresult, \pgfmathparse{\affcntry[\authorid]} \pgfmathresult
%       %     .
%     
    Correspondence concerning this article should be addressed to Molly
    Lewis, Psychology Department, Stanford University. 450 Serra Mall,
    Stanford, CA 94305. E-mail:
    \href{mailto:mll@stanford.edu}{\nolinkurl{mll@stanford.edu}}.

                                                                                    }

  \abstract{replicability, etc.}
  \keywords{replicability, reproducibility, meta-analysis, developmental psychology,
language acquisition \\

    \indent Word count: XXXX
  }

  \usepackage{setspace}
  \AtBeginEnvironment{tabular}{\singlespacing}
  \usepackage{pbox}

\begin{document}

\maketitle



\section{Introduction}\label{introduction}

Psychologists hope to build generalizable theories about human
behavior---theories that hold true beyond particulars of an individual
study. The field has grown concerned as a result in the face of recent
high-profile evidence that an effect observed in one study may not be
the same in another (``replicability crisis''; Ioannidis, 2005; Nosek,
2012, 2015). Some of this variability is to be expected, however---the
question we should instead be asking is, do the data provide support for
the theory, even if they are noisy? Furthermore, to build parsimonious
theories of human behavior, we should seek to explain not just
individual phenemenon, but entire literatures of research. What is
needed, then, is a tool for aggregating noisy data across studies within
a phenomenon, as well as a common language for comparing effects across
phenomenona.

Meta-analytic methods provide a powerful tool for doing just this. The
basic unit of meta-analysis---the effect size---provides an estimate of
the \emph{size} of an effect, as well as a measure of uncertainty around
this point estimate. With such a continuous measure of success, we can
apply the same reasoning we use to aggregate noisy measurements over
participants in a single study: By assuming each \emph{study}, rather
than participant, is sampled from a population, we can appeal to the
classical statistical framework to combine estimates of the effect size
for a given phenomenon.

This quantitative approach provides a rich tool kit for synthesizing
across literatures. By describing different phenomena using the same
unit of measurement, we are able to compare effects in different
domains. Rather than simply concluding that two effects are both
``real,'' we can ask more fine-grained questions: Is effect \emph{X}
bigger than effect \emph{Y}? Does a moderator influence effect \emph{X}
in the same way as effect \emph{Y}? This type of continuous analysis
supports building quantitative models, and specifying theories that are
more precise and constraining.

In addition to these theoretical motivations, there are practical
reasons for conducting a quantitative synthesis. When planning an
experiment, an estimate of the size of an effect on the basis of prior
literature can inform the sample size needed to achieve a desired level
of power. Meta-analytic estimates of effect sizes can also aid in design
choices: If a certain paradigm tends to have overall larger effect sizes
than another, the strategic researcher might select this paradigm in
order to maximize the power of a study.

In practice, however, the feasability of this meta-analytic approach
relies on the field's commitment to practices that facilitate cumulative
science. These practices apply to all stages of the research process. At
the stage of experimental planning, researchers must pre-specify
analytical descision to limit ``researcher'' degrees of freedom
(Simmons, 2011; Simonsohn, 2014a, 2014b, 2014c). At the stage of
completion, researchers should share a result regardless of its
significance (Rosenthal, 1979; Fanelli 2012). And, at the stage of
sharing, researchers must provide enough information about the method
for another lab to conduct a close replication. Critically,r eports must
also contain complete descriptions of both data and analytical decisions
so that effect sizes can be calculcated for the purposes of
meta-analysis,

In the present paper, we use meta-analytic methods to provide a
quantitative synthesis of an entire field of psychological research:
language acquisition. We think this field is a particularly informative
case study. It may be particularly vulnerable to false findings because
running children is expensive (Ioanndis, 2005), and thus:

\begin{itemize}
\itemsep1pt\parskip0pt\parsep0pt
\item
  sample sizes are small
\item
  replications difficult and rare
\item
  Recent attention about practices in developmental research Peterson
  (2016)
\end{itemize}

We have two goals:

\begin{itemize}
\itemsep1pt\parskip0pt\parsep0pt
\item
  Describe the state of the field in terms of its participation in
  practices that are prerequisites to cumulative science, and
  ultimately, a theoretical synthesis
\item
  Provide a preliminary theoretical synthesis of the field
\end{itemize}

Towards this end, we introduce
\href{http://metalab.stanford.edu/}{Metalab}.

\section{Method}\label{method}

We calculated estimates of effect sizes for 11 different phenomenena in
language acquisition. We selected these phenomena in order to describe
development at many different levels of the language hierarchy, from the
acquistion of prosody and phonemic contrasts, to gaze following in
linguistic interaction. This wide range of phenomena allowed us to
compare the course of development across different domains, as well as
explore questions about the interactive nature of language acquisition.

Estimates of effect size were based on journal reports of experimental
data. In total, our sample includes estimates from 258 papers, 938
different conditions and 11,628 participants.

The process for selecting papers from the literature differed by domain,
with some individual meta-analyses using more systematic approaches than
others. {[}Simulations here?{]} \renewcommand{\arraystretch}{1.5}

\begin{table}[h!]
        \footnotesize
        \begin{tabular}{lp{4cm} p{5cm}r}
            \textbf{Level} & \textbf{Phenomenon}                                                               & \textbf{Description}                                                                                 & \textbf{N papers (conditions)}                                                                                                                                               \\
            \hline
            Prosody        & IDS  preference  \newline  {\scriptsize (Dunst, Gorman, \& Hamby, 2012)}          & {\scriptsize  Looking times as a function of whether infant-directed vs. adult-directed speech is presented as stimulation.}      & 16 (50)     \\
            Sounds         & Phonotactic learning  \newline {\scriptsize (Cristia, in prep)}                   & {\scriptsize Infants' ability to learn phonotactic generalizations from a short exposure.  }                  & 15 (47)                               \\
            ~              & Vowel discrimination (native) \newline {\scriptsize (Tsuji \& Cristia, 2014)}     & {\scriptsize Discrimination of native-language vowels, including results from a variety of methods.  }         & 40 (167)             \\ 
            ~              & Vowel discrimination (non-native) \newline {\scriptsize (Tsuji \& Cristia, 2014)} & {\scriptsize Discrimination of non-native vowels, including results from a variety of methods.  }     & 21 (72)     \\
            Phonotactics   & Statistical sound learning  \newline {\scriptsize (Cristia, in prep)}             & {\scriptsize Infants' ability to learn sound categories from their acoustic distribution.   }  & 11 (40) \\ 
            Proto-words    & Word segmentation \newline {\scriptsize  (Bergmann \& Cristia, 2015) }            & {\scriptsize Recognition of familiarized words from running, natural speech using behavioral methods.  }                     & 67 (295)                                     \\
            Words     &   Mutual exclusivity \newline {\scriptsize (Lewis \& Frank, in prep)} &{\scriptsize  Mapping of novel words reflecting children's inference that novel words tend to refer to novel objects.}
            & 20 (60)             \\
            ~              & Concept-label advantage   \newline {\scriptsize (Lewis \& Long, unpublished)}     & {\scriptsize Infants' categorization judgments in the presence and absence of labels.    } & 16 (100) \\
            ~              & Online word recognition \newline {\scriptsize (Frank, Lewis, \& MacDonald, 2016)} & {\scriptsize Online word recognition of familiar words using two-alternative forced choice preferential looking.   }              & 12 (32)                         \\
            Communication  & Gaze following  \newline {\scriptsize  (Frank, Lewis, \& MacDonald, 2016)}        & {\scriptsize Gaze following using standard multi-alternative forced-choice paradigms.   }                       & 15 (45)                                           \\
            ~              & Pointing and vocabulary  \newline {\scriptsize (Colonnesi et al., 2010)}          & {\scriptsize Longitudinal correlations between declarative pointing and later vocabulary.  }               & 25 (30)                         \\ 
        \end{tabular}
    \end{table}

\section{Replicability of the field}\label{replicability-of-the-field}

Effect size can vary between studies for reasons unrelated to a
theoretical construct. One reason for this variability is the precision
of the effect size, which we can model based on the sample size of the
study. A remaining source variability, however, are biases introduced
directly by the experimenter, via publication bias (Fanelli, 2010;
Rosenthal, 1979; Rothstein, Sutton, \& Borenstein, 2006), analytical
flexibility (Simmons, Nelson, \& Simonsohn, 2011), reporting errors, or
even fraud. These biases are much more difficult to model, and may
therefore lead to large but unknown errors in estimates of the effect
size. If these types of practices are present in the literature,
estimates of effect size may be poor estimates of the true underlying
effect size, making it difficult to make theoretical progress. Below we
present analyses examining whether signatures of publication bias and
analytical flexibility are present in the language acquistion
literature. We find little evivdence of these biases.

\begin{table}[h!]
    \footnotesize
    \begin{tabular}{lrrrrrr}
        \textbf{Phenomenon}                                 & $\mathbf{\bar{d}}$ & \textbf{power} & \textbf{p-curve skew} & \textbf{funnel skew} & \textbf{fail-safe-N} \\                   
        \hline
        IDS  preference                  & 0.71 [0.53, 0.89]         & ~         &      & ~            & ~   \\ 
        Phonotactic learning              & 0.04 [-0.09, 0.16]          & ~         &      & ~            & ~         \\ 
        Vowel discrimintation (native)    & 0.6 [0.5, 0.71]           & ~         &      & ~            & ~             \\ 
        Vowel discrimination (non-native) & 0.66 [0.42, 0.9]           & ~         &      & ~            & ~           \\ 
        Statistical sound learning        & -0.14 [-0.27, -0.02]           & ~         &      & ~            & ~             \\ 
        Word segmentation                 & 0.19 [0.14, 0.24]           & ~         &      & ~            & ~             \\ 
        Mutual exclusivity               & 1 [0.68, 1.33]           & ~         &     & ~            & ~             \\ 
        Concept-label advantage           & 0.4 [0.29, 0.51]           & ~         &      & ~            & ~              \\ 
        Gaze following                    & 0.84 [0.26, 1.42]          & ~         &      & ~            & ~             \\ 
        Pointing and vocabulary           & 0.41 [0.32, 0.49]           & ~         &      & ~            & ~               \\ 
    \end{tabular}
\end{table}

\subsection{p-curves}\label{p-curves}

(Simonsohn, Nelson, \& Simmons, 2014a, 2014b; Simonsohn, Simmons, \&
Nelson, 2015) Across studies we should expect some variability in effect
size due to sampling error alone. But this variability in effect size
should be \emph{systematic}: There should be less variability around the
mean for more precise studies, as measured by sample size. The presence
of variability in effect sizes that is not accounted for sample size may
suggest publication bias in a literature.

bias introduced by meta-analysis in selection (second-order selection
bias)

\subsection{Funnel}\label{funnel}

\begin{itemize}
\itemsep1pt\parskip0pt\parsep0pt
\item
  Egger's regression test
\end{itemize}

\subsection{Orwin}\label{orwin}

\subsection{Power}\label{power}

Ioannidis and Trikalinos (2007)

TABLE WITH: Eggers regression, p-curve (stouffer), regular power,
p-power, orwin

\section{Theoretical Synthesis}\label{theoretical-synthesis}

OUTLINE

\subsection{Statistical Approach}\label{statistical-approach}

METAMETAPLOT

\section{Discussion}\label{discussion}

\paragraph{Author Contributions}\label{author-contributions}

\paragraph{Acknowledgments}\label{acknowledgments}

\newpage

\subsubsection{References}\label{references}

\setlength{\parindent}{-0.5in} \setlength{\leftskip}{0.5in}
\setlength{\parskip}{8pt}

Bergmann, C., \& Cristia, A. (2015). Development of infants'
segmentation of words from native speech: A meta-analytic approach.
\emph{Developmental Science}.

Dunst, C., Gorman, E., \& Hamby, D. (2012). Preference for
infant-directed speech in preverbal young children. \emph{Center for
Early Literacy Learning}, \emph{5}(1).

Fanelli, D. (2010). Positive Results Increase Down the Hierarchy of the
Sciences. \emph{PLoS ONE}, \emph{5}(4), e10068--10.

Frank, M. C., Lewis, M. L., \& MacDonald, K. (in press). A performance
model for early word learning. In \emph{Proceedings of the 38th Annual
Conference of the Cognitive Science Society}. Retrieved from
\url{http://langcog.stanford.edu/papers_new/frank-2016-underrev.pdf}

Lewis, M., \& Frank, M. C. (in prep). Multiple routes to disambituation.

Peterson, D. (2016). The Baby Factory: Difficult Research Objects,
Disciplinary Standards, and the Production of Statistical Significance.
\emph{Socius: Sociological Research for a Dynamic World}, \emph{2}(0),
1--10.

Rosenthal, R. (1979). The file drawer problem and tolerance for null
results. \emph{Psychological Bulletin}, \emph{86}(3), 638.

Rothstein, H. R., Sutton, A. J., \& Borenstein, M. (2006).
\emph{Publication bias in meta-analysis: Prevention, assessment and
adjustments}. John Wiley \&amp; Sons.

Simmons, Nelson, L. D., \& Simonsohn, U. (2011). False-Positive
Psychology: Undisclosed Flexibility in Data Collection and Analysis
Allows Presenting Anything as Significant. \emph{Psychological Science},
\emph{22}(11), 1359--1366.

Simonsohn, Nelson, L. D., \& Simmons, J. P. (2014a). p-Curve and Effect
Size: Correcting for Publication Bias Using Only Significant Results.
\emph{Perspectives on Psychological Science}, \emph{9}(6), 666--681.

Simonsohn, Nelson, L. D., \& Simmons, J. P. (2014b). P-curve: A key to
the file-drawer. \emph{Journal of Experimental Psychology: General},
\emph{143}(2), 534.

Simonsohn, Simmons, J. P., \& Nelson, L. D. (2015). Better p-curves.
\emph{Simonsohn, Uri, Joseph P. Simmons, and Leif D. Nelson
(Forthcoming),``Better P-Curves,'' Journal of Experimental Psychology:
General}.

Tsuji, S., \& Cristia, A. (2014). Perceptual attunement in vowels: A
meta-analysis. \emph{Developmental Psychobiology}, \emph{56}(2),
179--191.



\end{document}
