%% Documentclass:
\documentclass[manuscript]{stjour}

%% Manuscript, for double spaced, larger fonts
% \documentclass[manuscript]{stjour}
%% Only needed if you use `manuscript' option
\journalname{Open Mind}
%\journalname{Computational Psychiatry}


%%%%%%%%%%% Please supply information %%%%%%%%%%%%%%%%%%%%%%%%%

%% For Open Mind:
%% Supplementary Materials:
\supplementslinks{dx.doi.org/10.1098/rsif.2013.0969}

%% For Computational Psychiatry
\supportinginfo{dx.doi.org/10.7910/DVN/PQ6ILM}

%% If no conflicts, this command doesn't need to be used
%% \conflictsofinterest{}

%%%%%%%%%%% to be supplied by MIT Press, only %%%%%%%%%%%%%%%%%

\citation{Niyogi, R. K., Breton, Y.-A., Solomon,
R. B., Conover, K.,\\ Shizgal, P., Dayan, P. (2015).\\ 
Optimal indolence: a normative microscopic approach to work and leisure. Open Mind 1(1):
1−12.}

\received{20 October 2013}
\accepted{7 November 2013}
\published{26 January 2014}

%% DOI address:
\setdoi{10.1098/rsif.2013.0969}

%%%%%%%% End MIT Press commands %%%%%%%%%%

%%%%%%%%%%%%%%%%%%%%%%%%%%%%%%%%%%%%%%%%%%%%%%%%%%%%%%%%%%%%%%%
%% author definitions should be placed here:

%% example definition
\def\taupav{\tau_{\mathrm{Pav}}}

\begin{document}
\title{Title of Article}
\subtitle{Subtitle Here}

%% If shortened title for running head is needed so that the article title can fit
%%   in the running head, use [] argument, ie,
%%
%%   \title[Shortened Title for Running Head]{Title of Article}
%%   \subtitle{Subtitle Here}

%% Since we use \affil{} in the argument of the \author command, we
%% need to supply another version of the author names, without \affil{}
%% to be used for running heads:

\author[Author Names]
{Author Names with affiliations\thanks{Current address: MacMurdo
Sound, Antartica}\affil{1},
Another Name\affil{2}, Still another Name\affil{2},\\
\and Final Name\affil{1}}

\affiliation{1}{Department, Institution, City, Country}

%ie.
%\affiliation{1}{Gatsby Computational Neuroscience Unit, University
%College London, London, United Kingdom} 

\affiliation{2}{Another Department, Institution, City, Country}

%ie
%\affiliation{2}{Center for Studies in
%Behavioral Neurobiology, Concordia University, Montreal, Quebec,
%Canada}

\correspondingauthor{Author Name}{Corresponding author email address}

% ie,
%\correspondingauthor{Ritwik K. Niyogi}{ritwik.niyogi@gatsby.ucl.ac.uk}

\keywords{(a series of uncapitalized words, separated with commas)}

%ie
%\keywords{work, leisure, normative, microscopic,  reinforcement learning, economics}

\begin{abstract}
Abstract text here.
\end{abstract}



\section{Sample Section}
Text here. Text here. Text here. Text here.
Text here. Text here. Text here. Text here.
Text here. Text here. Text here. Text here.
Text here. Text here. Text here. Text here.

\subsection{Sample Subsection}
Text here. Text here. Text here. Text here.
Text here. Text here. Text here. Text here.
Text here. Text here. Text here. Text here.
Text here. Text here. Text here. Text here.

\subsubsection{Sample Subsubsection}
Text here. Text here. Text here. Text here.
Text here. Text here. Text here. Text here.
Text here. Text here. Text here. Text here.
Text here. Text here. Text here. Text here.


\newpage
\section{Track Changes}
These tracking commands allow copyeditors to mark text so
that the author can consider whether he/she wants to make
the change. In the manuscript version of the article, 
a list of changes will automatically appear at the end of the
article, as you see here.

The comments will be seen when
\verb+\documentclass[manuscript]{stjour}+
is used, but will disappear in the final version,
leaving only the suggested change.

Thus, if the author doesn't like the change, he/she should
delete it before resubmitting his/her manuscript.

These commands are used:
\begin{verbatim}
\added{}
\deleted{}
\replaced{}{}
\explained{}
\end{verbatim}

Here is some text to illustrate the track changes
commands, quotes (with apologies) from Albert Einstein:

\begin{quotation}
Two things are \replaced{almost infinite}{infinite}\explain{`almost'
detracts from the strength of the statement}: the universe and
human stupidity;  and I'm not sure about the universe.

The most beautiful thing we can experience is the mysterious. It is
the source of all true art and science. \added{-- Albert Einstein}

It is the \added{supreme} art of the teacher to awaken joy in creative
expression and knowledge.
\end{quotation}

\subsection{Using [] to add comments to appear in the List of Changes}

\begin{verbatim}
\added[]{}
\deleted[]{}
\replaced[]{}{}
\end{verbatim}

\begin{quotation}
Two things are \replaced[AH]{almost infinite}{infinite}\explain{`almost'
detracts from the strength of the statement}: the universe and
human stupidity;  and I'm not sure about the universe.

The most beautiful thing we can experience is the mysterious. It is
the source of all true art and science. \added[RB, April 7, 2016,
4:30pm]{-- Albert Einstein}

It is the \added[NWK]{supreme} art of the teacher to awaken joy in creative
expression and knowledge.
\end{quotation}
\subsection{Citatation Tests}
Citing, \citet{anderson}, \citep{anderson}, and \cite{anderson}.

\nocite{*}
\bibliography{bibsamp}


\end{document}
