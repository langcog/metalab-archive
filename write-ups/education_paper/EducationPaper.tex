\documentclass[english,floatsintext,man]{apa6}

\usepackage{amssymb,amsmath}
\usepackage{ifxetex,ifluatex}
\usepackage{fixltx2e} % provides \textsubscript
\ifnum 0\ifxetex 1\fi\ifluatex 1\fi=0 % if pdftex
  \usepackage[T1]{fontenc}
  \usepackage[utf8]{inputenc}
\else % if luatex or xelatex
  \ifxetex
    \usepackage{mathspec}
    \usepackage{xltxtra,xunicode}
  \else
    \usepackage{fontspec}
  \fi
  \defaultfontfeatures{Mapping=tex-text,Scale=MatchLowercase}
  \newcommand{\euro}{€}
\fi
% use upquote if available, for straight quotes in verbatim environments
\IfFileExists{upquote.sty}{\usepackage{upquote}}{}
% use microtype if available
\IfFileExists{microtype.sty}{\usepackage{microtype}}{}

% Table formatting
\usepackage{longtable, booktabs}
\usepackage{lscape}
% \usepackage[counterclockwise]{rotating}   % Landscape page setup for large tables
\usepackage{multirow}		% Table styling
\usepackage{tabularx}		% Control Column width
\usepackage[flushleft]{threeparttable}	% Allows for three part tables with a specified notes section
\usepackage{threeparttablex}            % Lets threeparttable work with longtable

% Create new environments so endfloat can handle them
\newenvironment{ltable}
  {\begin{landscape}\begin{center}\begin{threeparttable}}
  {\end{threeparttable}\end{center}\end{landscape}}

\newenvironment{lltable}
  {\begin{landscape}\begin{center}\begin{ThreePartTable}}
  {\end{ThreePartTable}\end{center}\end{landscape}}

\usepackage{ifthen} % Only add declarations when endfloat package is loaded
\ifthenelse{\equal{man}{\string jou}}{%
  \DeclareDelayedFloatFlavor{ThreePartTable}{table} % Make endfloat play with longtable
  \DeclareDelayedFloatFlavor{ltable}{table} % Make endfloat play with lscape
  \DeclareDelayedFloatFlavor{lltable}{table} % Make endfloat play with lscape & longtable
}{}%


% The following enables adjusting longtable caption width to table width
% Solution found at http://golatex.de/longtable-mit-caption-so-breit-wie-die-tabelle-t15767.html
\makeatletter
\newcommand\LastLTentrywidth{1em}
\newlength\longtablewidth
\setlength{\longtablewidth}{1in}
\newcommand\getlongtablewidth{%
 \begingroup
  \ifcsname LT@\roman{LT@tables}\endcsname
  \global\longtablewidth=0pt
  \renewcommand\LT@entry[2]{\global\advance\longtablewidth by ##2\relax\gdef\LastLTentrywidth{##2}}%
  \@nameuse{LT@\roman{LT@tables}}%
  \fi
\endgroup}


  \usepackage{graphicx}
  \makeatletter
  \def\maxwidth{\ifdim\Gin@nat@width>\linewidth\linewidth\else\Gin@nat@width\fi}
  \def\maxheight{\ifdim\Gin@nat@height>\textheight\textheight\else\Gin@nat@height\fi}
  \makeatother
  % Scale images if necessary, so that they will not overflow the page
  % margins by default, and it is still possible to overwrite the defaults
  % using explicit options in \includegraphics[width, height, ...]{}
  \setkeys{Gin}{width=\maxwidth,height=\maxheight,keepaspectratio}
\ifxetex
  \usepackage[setpagesize=false, % page size defined by xetex
              unicode=false, % unicode breaks when used with xetex
              xetex]{hyperref}
\else
  \usepackage[unicode=true]{hyperref}
\fi
\hypersetup{breaklinks=true,
            pdfauthor={},
            pdftitle={Building broad-shouldered giants: meta-analytic methods for reproducible research},
            colorlinks=true,
            citecolor=blue,
            urlcolor=blue,
            linkcolor=black,
            pdfborder={0 0 0}}
\urlstyle{same}  % don't use monospace font for urls

\setlength{\parindent}{0pt}
%\setlength{\parskip}{0pt plus 0pt minus 0pt}

\setlength{\emergencystretch}{3em}  % prevent overfull lines

\setcounter{secnumdepth}{0}
\ifxetex
  \usepackage{polyglossia}
  \setmainlanguage{}
\else
  \usepackage[english]{babel}
\fi

% Manuscript styling
\captionsetup{font=singlespacing,justification=justified}
\usepackage{csquotes}
\usepackage{upgreek}



\usepackage{tikz} % Variable definition to generate author note

% fix for \tightlist problem in pandoc 1.14
\providecommand{\tightlist}{%
  \setlength{\itemsep}{0pt}\setlength{\parskip}{0pt}}

% Essential manuscript parts
  \title{Building broad-shouldered giants: meta-analytic methods for reproducible
research}

  \shorttitle{MetaLab Education}


  \author{Christina Bergmann\textsuperscript{1}, Sho Tsuji\textsuperscript{1}, Page Piccinini\textsuperscript{2}, Molly Lewis\textsuperscript{3}, Mika Braginsky\textsuperscript{3}, Michael C. Frank\textsuperscript{3}, \& Alejandrina Cristia\textsuperscript{1}}

  \def\affdep{{"", "", "", "", "", "", ""}}%
  \def\affcity{{"", "", "", "", "", "", ""}}%

  \affiliation{
    \vspace{0.5cm}
          \textsuperscript{1} Laboratoire de Sciences Cognitives et Psycholinguistique, ENS\\
          \textsuperscript{2} NeuroPsychologie Interventionnelle, ENS\\
          \textsuperscript{3} Department Psychology, Stanford University  }

 % If no author_note is defined give only author information if available
    \authornote{
    \newcounter{author}
                                                                                    }
  
  \note{Correspondence concerning this article should be addressed to Christina
Bergmann, Laboratoire de Sciences Cognitives et Psycholinguistique, ENS.
29 Rue d'Ulm, 75005 Paris, France. E-mail:
\href{mailto:chbergma@gmail.com}{\nolinkurl{chbergma@gmail.com}}}

  \keywords{replicability, reproducibility, meta-analysis, developmental psychology,
language acquisition \\

    \indent Word count: XXXX
  }



  \usepackage{setspace}
  \usepackage{float}
  \usepackage{graphicx}
  \AtBeginEnvironment{tabular}{\singlespacing}
  \usepackage{pbox}

\begin{document}

\maketitle



\section{Introduction}\label{introduction}

Psychology has recently seen a \enquote{crisis of confidence}, as recent
findings challenge both the validity of key findings CITE-MANYLABS1 as
well as the general replicability rate of published findings
CITE-MANYLABS2. In this process, some practices have been discussed as
introducing bias and error, starting at data collection, over analysis
to publication, CITE-FALSEPOSPSYC,SCIUTOPIA1,2. In other words,
psychology is today facing the same issues that caused substantial
changes in research practices within the medical sciences a decade ago
CITE-\url{http://www.nejm.org/doi/full/10.1056/nejme048225}. The present
paper discusses to what extent these issues are present in developmental
psychology, provides a quantitative estimation of the prevalence certain
problematic practices, and discuss potential solutions.

\textbf{\emph{CB:This last sentence (together with the next section)
implies we will look at p-hacking, which we currently don't and I wonder
whether we should.}}

\subsection{Relevance of the confidence crisis for developmental
psychology}\label{relevance-of-the-confidence-crisis-for-developmental-psychology}

The problems underlying recent confidence crises are thought to be true
of empirical sciences at large, given the current reward structure
CITE-UTOPIA. Specifically, at present researchers are valued on the
basis of the quantity and (some measure of) impact of their
publications, and publication in a high-impact venue is dependent, among
other factors, on (a) the topic being \enquote{hot}; (b) the result
being surprising; and (c) the result being statistically significant.
One of the obvious consequences of this reward structure is that
replications and even conceptual extensions are less likely to be
undertaken (since they are not rewarded as much), and if they are, they
are unlikely to be published, particularly if they reveal a
non-significant result. Furthermore, modeling suggests that these issues
may be exacerbated depending on the characteristics of a given subfield
(Ioannidis, 2005), specifically in fields where both the underlying
effect sizes and typical sample sizes are small.

We believe that all of these descriptions are highly relevant to
developmental studies, particularly those focusing early and middle
childhood. Since the population under investigation is costly to recruit
and difficult test, there is a pressure towards small sample sizes.
Moreover, populations are intrinsically variable, possibly leading to
greater variance and smaller underlying effect sizes. At the confluence
of these two factors, one expects to find habitually underpowered
studies, which is problematic both when assessing whether an effect is
truly present or absent, and when estimating its magnitude.

There is a conceptually separate set of issues, which is nonetheless
relevant given the aforementioned warning signs for developmental
psychology. One of the habits that has become under attack recently
pertains flexibility in data collection and analysis, as arguably this
flexibility exacerbates the incidence of false positives
CITE-FALSEPOSPSYC. And yet as developmentalists we often \enquote{tweak}
our paradigms, and explore new ones, in the constant search for more
reliable and robust methods. This is because testing infants and
children is inherently difficult, as there are both methodological and
time constraints. Explicitly instructing participants to respond with a
defined behavior, for example, is not possible before the second or
third birthday, adn even then, such measures are not always
implementable. Further, emerging technologies, such as eye-tracking and
tablets, have been eagerly adopted CITE-TABET {[}?{]}. As a result of
all these factors, multiple ways to tap into the same phenomenon have
been developed, but it is an open question to what extent these lead to
comparable results. Moreover, although current discussions appear to
discourage flexibility, we believe a strong argument can be made for
continuous exploration and evaluation of alternative methods, given that
another responsibility we have as scientists is to gather data of the
highest quality and interpretability. In that sense, sticking to old
methods for the sake of comparability would be uneconomical.

The above description would lead one to believe that developmental
psychology, particularly that focusing early childhood, should be liable
to the issues that have been identified and discussed in other subfields
as leading to a crisis in confidence. In the present paper, we seek to
provide a more informed overview of a subset of questions that we have
identified as key issues.

\subsection{Key issues}\label{key-issues}

\subsubsection{Replication and
replicability}\label{replication-and-replicability}

Replicability is a core concept in the recent crisis, as exactly this
property of scientific studies (and potentially whole sub-fields) in
Psychology has turned out to be surprisingly problematic. We define the
concept here, as authors vary in their understanding of what constitutes
a replication. Replicating a study means (in the context of this paper)
conducting a conceptually similar experiment with new stimuli and in a
slightly different population but following the same procedure and
analyses (based on the published report), tapping into the same
phenomenon, and with the same outcome as peviously reported (allowing
for a margin of error). Being able to (repeatedly) successfully
replicate a study can be taken as an indicator that the phenomenon under
investigation is true and therefore that theories can be built on it. In
addition, varying populations and using different, yet comparable
stimuli, assesses generalizability across presumably irrelevant
dimensions. If an effect does not generalize across stimuli or
populations, it is possible that a previously unknown limitation has
been uncovered, which needs to be specified for future replications and
within all theories building on the general effect.

Replicability can be assessed when aggregating all studies that aim to
tap into a given phenomenon and assessing whether -- taking all evidence
together -- the effect is statistically different from 0. A next step
consists of comparing the direction and magnitude of initial reports and
replications.\\\textbf{\emph{The second bit is only there implicitly}}

\subsubsection{Sample size and statistical
power}\label{sample-size-and-statistical-power}

Underpowered studies, that is studies with a low probability to detect
an effect given it is present in the population, pose a problem for
branches of developmental studies that interpret both significant and
nonsignificant findings; for example when tracking the emergence of an
ability as children mature or when examining the boundary conditions of
an ability. This practice is problematic for two reasons: On one hand,
the null hypothesis, for example that two groups do not differ, is not
being tested, so it cannot be adopted based on a high p-value. Instead,
p-values can only support rejections of the null hypothesis with a
certainty that the data at hand are incompatible with it below a pre-set
threshold. On the other hand, even in the most rigorous study design and
execution, null results will occur ever so often; for example in a study
with 80\% power (a number typically deemed sufficient), every fifth
result will not reflect that there is a true effect present in the
population. Disentangling whether a non-significant finding indicates
the absence of a skill, random measurement noise, or the lack of
experimental power to detect this skill reliably and with statistical
support is impossible based on p-values.

A second problem emerges when underpowered studies yield significant
outcomes, as the effects reported in such cases will be over-estimating
the true effect. This makes appropriate planning for future research
which aims to build on this report more difficult, as sample sizes will
be too small, leading to null-results which do not speak to the
phenomenon under investigation. This poses a serious hindrance for work
building on seminal studies, including replications across languages and
extensions. However, aggregating over such null-results using a graded
estimate, i.e.~a standardized effect size, can reveal whether a
phenomenon is present in the population and correct for the initial
over-estimation. In short, even a true positive result is insufficient
in the quest for the truth when it is underpowered.

In developmental psychology, reasons for sample size decisions are
rarely reported, it is thus unnclear how (or if) decisions about sample
size are made before commencing data collection. Formal prospective
power calculations are as yet rare, especially those based on multiple
studies. Alternatively, it is possible that ressource limitations
determine sample size, as recruitment can be difficult and is very
costly.

To investigate the status quo, we first compute typical power across a
range of phenomena in early language acquisition, and explore which
effect sizes are detectable with sample sizes in the included studies.
Further, we investigate how researchers might determine sample sizes
(for example by following the first paper in a literature), and whether
they take into account sensitivity of methods used.

\subsubsection{Procedural variability}\label{procedural-variability}

Often there is more than one way to measure a specific construct.
Consider for example a measurement of preference, such as when trying to
establish that infants distinguish infant-directed from adult-directed
speech (IDS and ADS, respectively) and show a strong preference for the
former. This preference can be measured in a number of ways, as it is
something children bring to the lab. In the meta-analysis on IDS
preference there are 4 different methods, all aiming to pick up the very
same phenomenon, and this specific line of investigation is no exception
in developmental psychology.

We will assess in how far the different methods used to test the same
construct vary in their sensitivity. Further, taking possible ressource
limitations into account, we consider drop-out rates as a potential
measure of interest and discuss whether higher exclusion rates coincide
with more precise measures, yielding higher effect sizes.

\subsubsection{P-hacking}\label{p-hacking}

During data collection and analysis, a number of practices can inflate
the number of significant p-values, effectively rendering p-values and
the notion of statistical significance meaninglss {[}false pos psy{]}.
First, flexible stopping rules, including adding observations when the
test statistic is \enquote{promising} or stopping data collection when a
result is \enquote{significant} increases the likelihood to obtain a
\enquote{significant} outcome. Another form of p-hacking is measuring
several dependent variables and conducting multiple significance tests
with each variable and with a combination of the variables. *** Check
the next statement: imo correct *** In developmental research, this
problematic practice further encompasses computing several dependent
variables (such as mean scores, difference scores, percentages, and so
on) based on the same measured data as well as selectively excluding
trials and re-testing the new data. Next, multiple conditions that
selectively can be dropped from the final report increase the number of
significance tests. Finally, it is problematic to post hoc introduce
covariates, most prominently gender, and test for an interaction with
the main effect. Finally combining two or more of these strategies again
inflated the number of significant results. All these practices might
seem innocuous and geared towards \enquote{helping} an effect to emerge
that the researcher believes to be real.

A \enquote{symptom} of such practices is a distribution of p-values with
increased frequency just below the significance threshold. P-curves test
for this problem, but they come with some limitations and only consider
statistically significant reports.

*** TODO/ Question: test variety in outcome measures,such as longest
look vs looking time vs reaction time vs other?***

\subsubsection{Publication biases}\label{publication-biases}

*** Include? I think it's sufficiently covered in previous sections***

As mentioned in the general introduction, current incentives including
publication of data in a prestigious journal, are geared towards
surprising and statistically significant studies. However, even when an
effect is robust and tested with sufficiently high participant numbers,
null results are expected to occur. This becomes even more pressing in a
field with small effect sizes and low numbers of participants. For an
accurate estimate of the true effect it is crucial to have access to all
results, to avoid overestimations.

*** TODO: Add table with overview and remedies {[}?{]} ***

\textbar{}Problematic practices \textbar{}Underlying issue
\textbar{}Symptoms \textbar{}Solution \textbar{}
\textbar{}:----------------------\textbar{}:--------------------\textbar{}:------------\textbar{}:----------------\textbar{}
\textbar{}Small studies \textbar{}Effect size under
\textbar{}Unpredictable\textbar{}Prospective power\textbar{} \textbar{}
\textbar{}investigation unknown\textbar{}Outcomes,
low\textbar{}calculations \textbar{}

\section{Methods}\label{methods}

\subsection{Source data: MetaLab}\label{source-data-metalab}

\textbf{\emph{AC- Not all information provided below seems relevant to
the goal of documenting good/bad practices (e.g. \% female participants)
from the point of view of the reader. Shouldn't the goal of this section
be to provide the minimal relevant info for the reader to understand how
we can answer our questions?}} \textbf{\emph{ChB: tried to shorten, feel
free to streamline more, we will have extensive submat and have the
website}}

In this paper, we extract measures of interest from meta-analyses of
child language development. Meta-analyses are built on a collection of
standardized effect sizes on a single, well-defined phenomenon. By
accumulating effect sizes and weighting them by their reliability
(effectively the sample size), it is possible to compute an estimate of
the population effect, as well as its structured variance. By harnessing
data from hundreds of studies, we can quantify patterns important for
experimental practices. Furthermore, combining multiple meta-analysis --
each centered on a different research question -- allows us to assess
whether current practices differ across different topics.

Given that all 11 meta-analyses we discuss in this paper focus on
language acquisition in early childhood, our suggestions will be most
relevant to this subfield. We present our methods and results to
researchers on developmental psychology in general to encourage others
to build similar meta-meta-analyses, thus allowing them to explore the
state of their own subfields and to improve their practices if
necessary. The analyses in this paper are based on MetaLab, an online
collection of meta-analyses on early language development. Currently,
MetaLab contains 11 meta-analyses, but it is open to submissions and
updates. The present analyses thus are a snapshot; through dynamic
reports on the website, and by downloading the freely available data, it
is continuously possible to obtain the most recent results.

In MetaLab, parts of each meta-analysis are standardized to allow for
the computation of common effect size estimates and for analyses that
span across different phenomena. These standardized variables include
study descriptors (such as citation and peer review status), participant
characteristics (including mean age, native language), methodological
information (for example what dependent variable was measured), and
information necessary to compute effect sizes (number of participants,
if available means and standard deviations of the dependent measure,
otherwise test statistics, such as t-values or F scores).

MetaLab contains datasets that address phenomena ranging from
infant-directed speech preference to mutual exclusivity, sampled
opportunistically based on data collected with involvement of (some)
authors of this paper (n=9 datasets) or they were extracted from
previously conducted meta-analyses related to language development (n=2,
i.e. ({\textbf{???}}, ({\textbf{???}}))). In the former case, we
attempted to document as much detail as possible for each entered
experiment (note that a paper can contain many experiments). Detailed
descriptions of all phenomena covered by MetaLab, including which papers
and other sources have been considered, can be found on the companion
website at metalab.stanford.edu and in the supporting information.

**** To be discussed\textbf{\emph{ Further, a throughout investigation
into data quality within MetaLab, including publication biases, and a
meta-meta-analyses have been conducted based on the same data (Lewis,
2016). }}\textless{}-- AC: I think the question of publication biases
should ABSOLUTELY be targeted in this paper, we need to explain what all
this means for our field!!!! *** \textbf{\emph{ChB: Is that double
dipping or so?}}

\textbf{\emph{AC- now this just sounds like unsupported bashing, too bad
you had to remove the justification. Incidentally, I don't think it's
relevant in this paper that is on practices, and not on main effects}}

\textbf{\emph{ChB: Actually, we want to incentivice making and using MAs
here as well, that's for the recommendations part (too be written once
we are clear on the results section, as those should be in parallel)}}

Meta-analyses do not rely on one (possibly inaccurate) study outcome, be
it significant or not. Despite their overall utility, meta-analyses are
not frequently conducted in most branches of developmental psychology.
Instead, narrative summaries are the dominant tool to build theories,
and that single studies are cited as evidence for the presence or
absence of an ability instead of meta-analyses.

\subsection{Statistical approach}\label{statistical-approach}

As dependent measure, we report Cohen's \emph{d}, a standardized effect
size based on comparing sample means and their variance. This effect
size was calculated when possible from means and standard deviations
across designs with the appropriate formula. When these data were not
available, we used test statistics, more precisely t-values or F scores
of the test assessing the main hypothesis. We also computed effect size
variance, which allows to weigh each effect size when aggregating across
studies. The variance is mainly determined by the number of
participants; intuitively effect sizes based on larger samples will be
weighted higher. Note that for research designs testing participants in
two conditions that need to be compared (for example exposing the same
infants to infant- and adult-directed speech), correlations between
those two measures are needed to estimate the effect size variance. This
measure is usually not reported, despite being necessary for effect size
calculation. Some correlations could be obtained through direct contact
with the original authors (see e.g., (Bergmann \& Cristia, 2015) for
details), for others we estimated this factor based on the information
in our database.

To aggregate effect sizes within a phenomenon, we used a multilevel
approach, which takes into account not only the effect sizes and their
variance of single studies, but also that effect sizes from the same
paper will be based on more similar studies than effect sizes from
different papers (Konstantopoulos, 2011), implemented in the metafor
package (Viechtbauer, 2010) of R (R Core Team, 2016). We excluded as
outliers effect sizes that were more than three standard deviations away
from the median effect size within each dataset, thus accounting for the
difference in median effect size across phenomena.

\section{Results and discussion}\label{results-and-discussion}

\subsection{Measures of a typical study on early language
acquisition}\label{measures-of-a-typical-study-on-early-language-acquisition}

Table 1 provides a summary of typical sample sizes and effect sizes by
phenomenon, but before discussing those descriptors in detail, we begin
by characterizing the overall snapshot provided by our data. Overall, we
observe small sample sizes (the overall median in our dataset is 18).
With such a sample size, and assuming a paired t-test based on
within-participant comparisons (the most frequent experiment design and
test statistic) it is possible to detect an effect in 80\% of all
studies when Cohen's \emph{d} = 0.70, in other words when investigating
a medium to large effect. When comparing groups, this number increases
to Cohen's \emph{d} = 0.96, a large effect.

The above observation about sample sizes and which effect size could be
detected in a typical study on early language acquisition is in stark
contrast with the effect sizes we actually observe, which tend to fall
into ranges of small to medium effects. Taking a closer look at single
phenomena, which we characterize along a number of dimensions, such as
typical age and the number of studies (and papers) we base our
observations on. Each effect size was calculated with a hierarchical
random effects model (Viechtbauer, 2010), taking into account increased
similarity between studies in the same paper and weighting effect sizes
by their variance (driven by the number of participants). Based on the
meta-analytical effect size and the median number of participants, we
calculated typical power (using the pwr package (Champely, 2015)). We
remind the reader that recommendations are for this value to be above
80\%, which refers to a likelihood that 4 out of 5 studies show a
significant outcome for an effect truly present in the population. It
turns out that by and large, studies are underpowered.

*** Question: Is the next bit over-interpreting our data? ***

Phenomena in MetaLab differ in the age groups typically tested and the
age range covered, with the mean age ranging between 4.5 months (infant
directed speech preference) and 2.5 years (mutual exclusivity). One
might expect a relationship between effect sizes and infant age both for
theoretical and practical reasons. On one hand, younger infants might
show a smaller effect in general because they are not yet as proficient
in their native language, having had less experience, and because they
are a more immature in terms of their information processing abilities
{[}CITE{]}. On the practical side, methods -- a topic we will
investigate in depth in the next section -- might be more noisy for
younger infants and they could be a more difficult population to
recruit.

While there is no strict linear relationship between infant age and
sample size, effect size, and the derived power, we observe a difference
between studies typically testing infants younger than one year and
those testing older infants. First, sample sizes are much lower for
younger infants, which do usually not test more than 20 infants
(although all datasets contain studies with larger samples). This is not
the case for older children. The only exception is the dataset
addressing mutual exclusivity, which habitually tests around 16
children. This low number of participants, however, is at least somewhat
off-set by a comparatively large effect size. Additionally, the number
of participants tested within each dataset ranges a great deal, between
single-digit numbers and in some cases more the tenfold amount. This
might indicate that researchers are mostly limited by their resources
and participant availability in planning their studies.

Turning to effect size, we see a similar split by age group in our data.
Younger infants show both a greater range and include lower effect sizes
which fall into the classical range of small effects (Cohen's \emph{d}
below .5), which is not the case for older children. Power is directly
related to sample size and effect size, so it is not surprising that
typical power is greater for older children. Interestingly, however,
there seems to be little to no relationship between effect sizes and
number of participants typically tested. For phenomena with large
effects, this means that studies are very high-powered (see gaze
following, online word recognition, as two examples). For younger
children, because sample sizes and effect sizes are both small, power is
habitually very low, and the only dataset which typically achieves
appropriate power near 80\% is non-native vowel discrimination. For
older children, power is solely caused by lower effect sizes. The lack
of a relationship between overall meta-analytic power and sample size
might indicate that researchers' experiment planning is not impacted by
the phenomenon under investigation. Studies might instead be designed
and conducted with pragmatic considerations in mind, such as participant
availability.

Besides this very general point, we refrain here from strong conclusions
based on the above-discussed observations, since the present dataset is
not exhaustive and topics typically investigated in younger children are
over-represented. However, we sampled in an opportunistic and thus to
some degree random fashion and the phenomena covered span very different
aspects of language acquisition and linguistic processing.

\begin{longtable}[c]{@{}lrrrrrrrrr@{}}
\caption{Descriptions of meta-analyses currently in
MetaLab.}\tabularnewline
\toprule
Topic & Mean Age (Months) & Median Sample Size & Min. Sample Size & Max.
Sample Size & \# Effect Sizes & \# Papers & d & SE & Avg
Power\tabularnewline
\midrule
\endfirsthead
\toprule
Topic & Mean Age (Months) & Median Sample Size & Min. Sample Size & Max.
Sample Size & \# Effect Sizes & \# Papers & d & SE & Avg
Power\tabularnewline
\midrule
\endhead
Infant directed speech preference & 4.34 & 20.00 & 10 & 60 & 48 & 16 &
0.73 & 0.13 & 0.61\tabularnewline
Vowel discrimination (native) & 6.54 & 12.00 & 6 & 50 & 112 & 29 & 0.69
& 0.09 & 0.37\tabularnewline
Sound symbolism & 6.77 & 20.00 & 11 & 26 & 42 & 10 & 0.25 & 0.11 &
0.12\tabularnewline
Vowel discrimination (non-native) & 7.69 & 16.00 & 8 & 30 & 46 & 14 &
0.79 & 0.24 & 0.58\tabularnewline
Statistical sound category learning & 8.16 & 14.75 & 5 & 35 & 16 & 9 &
-0.26 & 0.16 & 0.10\tabularnewline
Word segmentation & 8.25 & 20.00 & 4 & 64 & 276 & 65 & 0.16 & 0.03 &
0.08\tabularnewline
Phonotactic learning & 10.69 & 18.00 & 8 & 40 & 47 & 15 & 0.12 & 0.07 &
0.06\tabularnewline
Label advantage in concept learning & 11.96 & 13.00 & 9 & 32 & 99 & 17 &
0.34 & 0.05 & 0.13\tabularnewline
Gaze following & 14.24 & 23.00 & 12 & 63 & 32 & 11 & 1.08 & 0.16 &
0.95\tabularnewline
Online word recognition & 18.00 & 25.00 & 16 & 95 & 14 & 6 & 1.24 & 0.26
& 0.99\tabularnewline
Mutual exclusivity & 23.99 & 16.00 & 8 & 72 & 58 & 19 & 0.81 & 0.14 &
0.61\tabularnewline
\bottomrule
\end{longtable}

\textbf{\emph{TODO: Visualize power / Question: How?}}

\subsubsection{Comparing meta-analytic effect size and oldest paper to
estimate
power}\label{comparing-meta-analytic-effect-size-and-oldest-paper-to-estimate-power}

As Table 1 shows, experimenters are habitually not including a
sufficient number of participants to observe a given effect, assuming
the meta-analytic estimate is accurate. It might, however, be possible,
that power has been determined based on a seminal paper to be replicated
and/or built on. Initial reports tend to overestimate effect sizes
(Jennions \& M{ø}ller, 2002), possibly explaining the lack of power in
some datasets. We extracted for each dataset the oldest paper and
therein the largest reported effect size and re-calculated power
accordingly, using again the median sample size. The results are shown
in the table below. It turns out that in some cases, such as native and
non-native vowel discrimination, sample size choices match well with the
oldest report. The difference in power, noted in the last column, can be
substantial, with native vowel discrimination and phonotactic learning
being the two most salient examples. Here, sample sizes match well with
the oldest report and studies would be appropriately powered if this
estimate were representative of the true effect.

\begin{longtable}[c]{@{}llrrrr@{}}
\caption{For each meta-analysis, largest d from oldest paper and power,
along with the difference between power based on meta-analytic and
oldest d.}\tabularnewline
\toprule
Meta-analysis (MA) & Oldest Paper & Oldest d & Median Sample Size &
Power & Improvement\tabularnewline
\midrule
\endfirsthead
\toprule
Meta-analysis (MA) & Oldest Paper & Oldest d & Median Sample Size &
Power & Improvement\tabularnewline
\midrule
\endhead
Statistical sound category learning & Maye, Werker, \& Gerken (2002) &
0.56 & 14.75 & 0.31 & 0.21\tabularnewline
Word segmentation & Jusczyk \& Aslin (1995) & 0.56 & 20.00 & 0.40 &
0.33\tabularnewline
Mutual exclusivity & Merriman et al. (1989) & 0.70 & 16.00 & 0.48 &
-0.13\tabularnewline
Label advantage in concept learning & Balaban \& Waxman (1997) & 0.86 &
13.00 & 0.56 & 0.42\tabularnewline
Vowel discrimination (non-native) & Trehub (1976) & 1.02 & 16.00 & 0.80
& 0.22\tabularnewline
Phonotactic learning & Chambers et al. (2003) & 0.98 & 18.00 & 0.81 &
0.75\tabularnewline
Sound symbolism & Maurer, Pathman, \& Mondloch (2006) & 0.95 & 20.00 &
0.84 & 0.71\tabularnewline
Online word recognition & Zangl et al. (2005) & 0.89 & 25.00 & 0.87 &
-0.12\tabularnewline
Gaze following & Mundy \& Gomes (1998) & 1.29 & 23.00 & 0.99 &
0.04\tabularnewline
Vowel discrimination (native) & Trehub (1973) & 1.87 & 12.00 & 0.99 &
0.63\tabularnewline
Infant directed speech preference & Glenn \& Cunningham (1983) & 2.39 &
20.00 & 1.00 & 0.39\tabularnewline
\bottomrule
\end{longtable}

\begin{figure}[htbp]
\centering
\includegraphics{EducationPaper_files/figure-latex/Plot of Difference of d Values-1.pdf}
\caption{Correlation of largest d from oldest paper and difference
between oldest d and meta-analytic d.}
\end{figure}

To illustrate the disparity between the oldest effect size and the
meta-analytic effect, and consequently the difference in power, we plot
the difference between both against the oldest effect. This difference
is larger as oldest effect size increases, with an average of 0.51
compared with an average effect size of 0.59 (note that we based this on
the absolute value). The plot showcases that researchers might want to
be wary of large effects, as they are more likely to be
non-representative of the true phenomenon compared to smaller initial
effects being reported. Especially when making decisions about sample
sizes, large effect might thus not be the best guide. Taking the
above-mentioned mean values as example, a realistic sample size to
ensure 80\% power would be 46.23 participants, instead of 14.07
participants suggested by the first paper. While these numbers average
over research questions and methods, which all influence the specific
number of participants necessary, this example showcases that
experimenters should take into account as much evidence as available to
be able to plan for robust and reproducible studies.

\subsubsection{Power over time}\label{power-over-time}

The comparison of initial and meta-analytic effect size has a number of
caveats, for example, as we will lay out in the next section, methods
might be different between initial reports and our overall sample; the
availability of methods changes over time, as new approaches are being
developed and automated procedures become more common. Further, the
largest effect size from a seminal paper might have been spurious, and
the research community could well be aware of that. In additional, as
infant research becomes more common, recruitment and obtaining funds
might both become easier, thereby increasing typical sample size over
the years. For a more continuous approach, we thus investigate power
(which is determined by effect size and sample size) as follows. We
first generate a meta-analytic model for each dataset that takes into
account infant age and method and then derive the respective to be
expected effect size base on those data for each entry in this dataset.
Power is then estimated based on the sample size actually tested.

Across datasets we observe a general negative trend, with its steepness
varyig avross dataset. The only positive trends occur in the upper
ranges of estimated power and for older children.

*** Add more here? ***

\begin{figure}[htbp]
\centering
\includegraphics{EducationPaper_files/figure-latex/Plot of power over time-1.pdf}
\caption{To BE UPDATED: Summary plot of power across years and
meta-analyses}
\end{figure}

\subsection{Procedure comparison}\label{procedure-comparison}

In this section we address how methods might be chosen, adopting two
angles. We first take a pragmatic, resource-oriented approach and
compare methods with respect to their dropout rate. Then we compare how
effect size across phenomena is relating to method choice.

\subsubsection{Drop-out rates across methods and
age}\label{drop-out-rates-across-methods-and-age}

Choosing a robust method can help increase the power of studies, such
that more precise measurements lead to larger effects and thus require
fewer participants to be tested. However, the number of participants
relates to the final sample and not how many infants had to be invited
into the lab. We thus first quantify whether methods differ in their
typical drop-out rate, as the available participant pool might inform
method choice. To this end we consider all methods across datasets in
MetaLab which have more than 10 associated effect sizes and for which
information on the number of dropouts was reported; this information is
not always available in the published report, and in the case of the two
meta-analyses we added based on published reports, the information was
not added. Therefore, the following analyses only cover 4 methods and
172 data points.

The results of the linear mixed effect model predicting dropout rate by
method and mean participant age (while accounting for the different
effects being tested) are summarized in the table below. The results
show that, taking central fixation as baseline, conditioned headturn and
stimulus alternation have significantly more drop-outs. Figure XXX
underlines this observation, and illustrates the relationship of
drop-out rate with age. Overall, stimulus alternation leads to the
highest drop-out rates, which lies at around 50\% across all age groups.
While age is not significantly impacting drop-out rates, it interacts
with the different methods. We observe an increase in drop-out rates,
which is most prominent in conditioned headturn (a significant
interaction) and headturn preference procedure (where the interaction
approaches significance).

Interestingly, the methods with lower drop-out rates, namely central
fixation and headturn preference procedure, are among the most frequent
ones in MetaLab and certainly more frequent than those with higher
drop-out rates, indicating that drop-out rate might inform researchers'
choices. Being able to retain more participants as a factor in method
choice points to the mentioned limitations regarding the participant
pool we mentioned before, as more participants will have to be tested to
arrive at the same sample size.

*** Question: method by total participants run (aka
resource-intensity)?***

\begin{longtable}[c]{@{}lrrr@{}}
\caption{Method vs Dropout}\tabularnewline
\toprule
& Estimate & Std. Error & t value\tabularnewline
\midrule
\endfirsthead
\toprule
& Estimate & Std. Error & t value\tabularnewline
\midrule
\endhead
(Intercept) & 28.41 & 2.20 & 12.90\tabularnewline
methodconditioned head-turn & 30.20 & 5.47 & 5.52\tabularnewline
methodhead-turn preference procedure & -2.13 & 3.22 &
-0.66\tabularnewline
methodstimulus alternation & 21.09 & 3.97 & 5.31\tabularnewline
ageC & 0.27 & 0.47 & 0.59\tabularnewline
methodconditioned head-turn:ageC & 2.96 & 1.18 & 2.51\tabularnewline
methodhead-turn preference procedure:ageC & 1.11 & 0.72 &
1.54\tabularnewline
methodstimulus alternation:ageC & -0.17 & 0.92 & -0.18\tabularnewline
\bottomrule
\end{longtable}

\begin{figure}[htbp]
\centering
\includegraphics{EducationPaper_files/figure-latex/Plot of Effect of Method on Dropout Rate-1.pdf}
\caption{Percent dropout as explained by different methods and mean age
of participants.}
\end{figure}

\subsubsection{The effect of method choice on effect sizes (and thus
power)}\label{the-effect-of-method-choice-on-effect-sizes-and-thus-power}

Methods which retain a lot of participants might either be more suitable
to test infants, decreasing noise as most participants are on task, or
less selective, thus increasing noise as participants who for example
are fussy are more likely to enter the data pool. We operationalize
precision as the size of the effect measured. Some datasets contain only
one method, making it thus difficult to disentangle the effect size of a
phenomenon with the change of effect size introduced by different
methods. To avoid this confound, we limited this investigation to the 4
datasets that contain three or more different methods. We further only
investigate those methods that have at least 10 effect sizes in our
overall dataset. Thus, the present analyses are limited to 232
observations.

Table: Effect of d by method with central fixation as baseline method.

We built a meta-analytic model with the effect size measure Cohen's
\emph{d} as the dependent variable, method and mean age centered as
independent variables. The model also includes the variance of \emph{d}
for sampling variance, and paper within meta-analysis as a random effect
(because we assume that within a paper experiments and thus effect sizes
will be more similar to each other than across papers). Since the model
compares one method as the baseline to all other methods, a baseline
method had to be chosen. \enquote{Central fixation} was included as the
baseline method, as it appears most frequently in the 4 datasets
included in this analysis (100 times out of 232 total entries of the
selected meta-analyses).

\begin{figure}[htbp]
\centering
\includegraphics{EducationPaper_files/figure-latex/Plot of Effect of Method-1.pdf}
\caption{Effect size as explained by different methods and mean age of
participants.}
\end{figure}

The model results in Table XXX show that compared to central fixation
only conditioned headturn yields reliably higher effect sizes, all other
methods do not statistically differ from this baseline. When factoring
in age, no interaction reaches significance, while this factor on its
own is marginally below the significance threshold, indicating that as
infants mature effect sizes increase across methods -- an observation
consistent with the view that infants and toddlers become more
proficient language users and are increasingly able to react
appropriately in the lab.

Comparing our analyses (Table XXX) and Figure YY in this section with
those in the previous section, it seems that high drop-out rates might
be offset by high effect sizes in the case of conditioned headturn.
While drop-out rates are around 40-50\%, effect sizes are above 1. Only
high-amplitude sucking seems to generate even higher effect sizes, but
for this method we did not have enough information on drop-out rates
available, so we cannot examine the relationship between the two.
Further, due to the few data points available (13 associated effect
sizes) the difference between high-amplitude sucking and central
fixation was not significant. Stimulus alternation does not fall into
this pattern of high drop-out rates being correlated with high effect
sizes, as the observed outcomes are in the range typical for
meta-analyses in our dataset.

There is an important caveat to this interpretation that some methods,
specifically conditioned headturn, which have higher dropout rates are
better at generating high effect sizes due to decreased noise (e.g., by
excluding infants that are not on task). Studies with fewer participants
(thanks to higher drop-out rates) might simply be underpowered, and thus
any significant finding is likely to over-estimate the effect. Due to
publication biases, we might not have access to all null results using
the same method, and thus the overestimation is directly reflected in
our effect size estimate. We take this caveat into account and compare

\subsection{P-hacking and publication
biases}\label{p-hacking-and-publication-biases}

*** Note: Previously not part of this paper. Possible analyses include
funnel plot asymmetry, p-curves. See synthesis paper. ***

\subsection{Working towards
cumulativity}\label{working-towards-cumulativity}

*** ChB: Alex, what do you envision here?***

\section{Summary and suggestions for the
future}\label{summary-and-suggestions-for-the-future}

We set out to discuss potentially problematic practices. We see X Y and
Z. We are thus adding to a recently emerging literature that critically
examines long-held standards and practices in order to make the whole
field more reliable and robust (Mills-Smith, Spangler, Panneton, \&
Fritz, 2015, Csibra, Hernik, Mascaro, Tatone, \& Lengyel (2016)).

\subsection{Concrete recommendations for developmental
psychologists}\label{concrete-recommendations-for-developmental-psychologists}

*** Check table to be made in the beginning ****

\subsubsection{1. Calculate power
prospectively}\label{calculate-power-prospectively}

\subsubsection{2.}\label{section}

\subsubsection{Increase availability and use of
meta-analyses}\label{increase-availability-and-use-of-meta-analyses}

To support the improvement current practices, we propose to make
meta-analyses available in the form of ready-to-use online tools,
dynamic reports, and as raw data. These different levels allow
researchers with varying interest and expertise interests to make the
best use of the extant record on infant language development, including
study planning by choosing robust methods and appropriate sample sizes.
There are additional advantages for interpreting single results and for
theory building that emerge from our dataset. On one hand, researchers
can easily check whether their study result falls within the expected
range of outcomes for their research question -- indicating whether or
not a potential moderator influenced the result. On the other hand,
aggregating over many data points allows for the tracing of emerging
abilities over time, quantifying their growth, and identifying possible
trajectories and dependencies across phenomena (for a demonstration see
({\textbf{???}})). Finally, by making our data and source code open, we
also invite contributions and can update our data, be it by adding new
results, filedrawer studies, or new datasets. Our implementation of this
proposal is freely online available at metalab.stanford.edu.

We have shown that power varies greatly across phenomena and that method
choice is important. It turns out, however, that researchers do not
choose the most robust methods. This might to be due to a lack of
consideration of meta-analytic effect size estimates. One of the reasons
for this is a lack of information on and experience in how to interpret
effect size estimates and use them for study planning {[}cite infancy
paper{]}. Meta-analyses on infant language development are also rare, as
showcased by the fact that the present dataset relied on the authors'
involvement, and only two out of 12 meta-analyses used could be
extracted from the extant work, and an extensive search in the present
literature did not yield additional candidates (excluding clinical
contexts). Conducting a meta-analysis is a laborious process,
particularly according to common practice where only a few people do the
work, with little support tools and educational materials available.
Incentives for creating meta-analyses are low, as public recognition is
tied to a single publication. The benefits of meta-analyses for the
field, for instance the possibility to conduct power analyses, are often
neither evident nor accessible to individual researchers, as the data
are not shared and traditional meta-analyses remain static after
publication, aging quickly as new results emerge.

A second possible reason lies in the availability of data allowing for
the conclusions we were able to draw here, be it in the form of reported
effect sizes within paper or as ready-to-use dataset. As noted
elsewhere, researchers do not report effect sizes ({\textbf{???}}),
despite long-standing recommendations to move beyond the persistent
focus on p-values (eg. APA Recommendations). A final impediment to
meta-analyses in developmental science are current reporting standards,
which make it difficult and at times even impossible to compute effect
sizes from the published literature.

\section*{References}\label{references}
\addcontentsline{toc}{section}{References}

Bergmann, C., \& Cristia, A. (2015). Development of infants'
segmentation of words from native speech: A meta-analytic approach.
\emph{Developmental Science}.

Champely, S. (2015). \emph{pwr: Basic Functions for Power Analysis}.
Retrieved from \url{https://CRAN.R-project.org/package=pwr}

Csibra, G., Hernik, M., Mascaro, O., Tatone, D., \& Lengyel, M. (2016).
Statistical treatment of looking-time data. \emph{Developmental
Psychology}, \emph{52}(4), 521--536.

Ioannidis, J. P. (2005). Why most published research findings are false.
\emph{PLoS Med}, \emph{2}(8), e124.

Jennions, M. D., \& M{ø}ller, A. P. (2002). Relationships fade with
time: A meta-analysis of temporal trends in publication in ecology and
evolution. \emph{Proceedings of the Royal Society of London B:
Biological Sciences}, \emph{269}(1486), 43--48.

Konstantopoulos, S. (2011). Fixed effects and variance components
estimation in three-level meta-analysis. \emph{Research Synthesis
Methods}, \emph{2}(1), 61--76.

Lewis, B., M. (2016). A Quantitative Synthesis of Early Language
Acquisition Using Meta-Analysis. \emph{Preprint}. Retrieved from
\url{https://osf.io/htsjm/}

Mills-Smith, L., Spangler, D. P., Panneton, R., \& Fritz, M. S. (2015).
A missed opportunity for clarity: Problems in the reporting of effect
size estimates in infant developmental science. \emph{Infancy},
\emph{20}(4), 416--432.

R Core Team. (2016). \emph{R: A language and environment for statistical
computing}. Vienna, Austria: R Foundation for Statistical Computing.
Retrieved from \url{https://www.R-project.org/}

Viechtbauer, W. (2010). Conducting meta-analyses in R with the metafor
package. \emph{Journal of Statistical Software}, \emph{36}(3), 1--48.
Retrieved from \url{http://www.jstatsoft.org/v36/i03/}





\end{document}
