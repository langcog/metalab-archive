\documentclass[english,floatsintext,man]{apa6}

\usepackage{amssymb,amsmath}
\usepackage{ifxetex,ifluatex}
\usepackage{fixltx2e} % provides \textsubscript
\ifnum 0\ifxetex 1\fi\ifluatex 1\fi=0 % if pdftex
  \usepackage[T1]{fontenc}
  \usepackage[utf8]{inputenc}
\else % if luatex or xelatex
  \ifxetex
    \usepackage{mathspec}
    \usepackage{xltxtra,xunicode}
  \else
    \usepackage{fontspec}
  \fi
  \defaultfontfeatures{Mapping=tex-text,Scale=MatchLowercase}
  \newcommand{\euro}{€}
\fi
% use upquote if available, for straight quotes in verbatim environments
\IfFileExists{upquote.sty}{\usepackage{upquote}}{}
% use microtype if available
\IfFileExists{microtype.sty}{\usepackage{microtype}}{}

% Table formatting
\usepackage{longtable, booktabs}
\usepackage{lscape}
% \usepackage[counterclockwise]{rotating}   % Landscape page setup for large tables
\usepackage{multirow}		% Table styling
\usepackage{tabularx}		% Control Column width
\usepackage[flushleft]{threeparttable}	% Allows for three part tables with a specified notes section
\usepackage{threeparttablex}            % Lets threeparttable work with longtable

% Create new environments so endfloat can handle them
\newenvironment{ltable}
  {\begin{landscape}\begin{center}\begin{threeparttable}}
  {\end{threeparttable}\end{center}\end{landscape}}

\newenvironment{lltable}
  {\begin{landscape}\begin{center}\begin{ThreePartTable}}
  {\end{ThreePartTable}\end{center}\end{landscape}}

\usepackage{ifthen} % Only add declarations when endfloat package is loaded
\ifthenelse{\equal{man}{\string jou}}{%
  \DeclareDelayedFloatFlavor{ThreePartTable}{table} % Make endfloat play with longtable
  \DeclareDelayedFloatFlavor{ltable}{table} % Make endfloat play with lscape
  \DeclareDelayedFloatFlavor{lltable}{table} % Make endfloat play with lscape & longtable
}{}%


% The following enables adjusting longtable caption width to table width
% Solution found at http://golatex.de/longtable-mit-caption-so-breit-wie-die-tabelle-t15767.html
\makeatletter
\newcommand\LastLTentrywidth{1em}
\newlength\longtablewidth
\setlength{\longtablewidth}{1in}
\newcommand\getlongtablewidth{%
 \begingroup
  \ifcsname LT@\roman{LT@tables}\endcsname
  \global\longtablewidth=0pt
  \renewcommand\LT@entry[2]{\global\advance\longtablewidth by ##2\relax\gdef\LastLTentrywidth{##2}}%
  \@nameuse{LT@\roman{LT@tables}}%
  \fi
\endgroup}


  \usepackage{graphicx}
  \makeatletter
  \def\maxwidth{\ifdim\Gin@nat@width>\linewidth\linewidth\else\Gin@nat@width\fi}
  \def\maxheight{\ifdim\Gin@nat@height>\textheight\textheight\else\Gin@nat@height\fi}
  \makeatother
  % Scale images if necessary, so that they will not overflow the page
  % margins by default, and it is still possible to overwrite the defaults
  % using explicit options in \includegraphics[width, height, ...]{}
  \setkeys{Gin}{width=\maxwidth,height=\maxheight,keepaspectratio}
\ifxetex
  \usepackage[setpagesize=false, % page size defined by xetex
              unicode=false, % unicode breaks when used with xetex
              xetex]{hyperref}
\else
  \usepackage[unicode=true]{hyperref}
\fi
\hypersetup{breaklinks=true,
            pdfauthor={},
            pdftitle={Building broad-shouldered giants: meta-analytic methods for reproducible research},
            colorlinks=true,
            citecolor=blue,
            urlcolor=blue,
            linkcolor=black,
            pdfborder={0 0 0}}
\urlstyle{same}  % don't use monospace font for urls

\setlength{\parindent}{0pt}
%\setlength{\parskip}{0pt plus 0pt minus 0pt}

\setlength{\emergencystretch}{3em}  % prevent overfull lines

\setcounter{secnumdepth}{0}
\ifxetex
  \usepackage{polyglossia}
  \setmainlanguage{}
\else
  \usepackage[english]{babel}
\fi

% Manuscript styling
\captionsetup{font=singlespacing,justification=justified}
\usepackage{csquotes}
\usepackage{upgreek}



\usepackage{tikz} % Variable definition to generate author note

% fix for \tightlist problem in pandoc 1.14
\providecommand{\tightlist}{%
  \setlength{\itemsep}{0pt}\setlength{\parskip}{0pt}}

% Essential manuscript parts
  \title{Building broad-shouldered giants: meta-analytic methods for reproducible
research}

  \shorttitle{MetaLab Education}


  \author{Christina Bergmann\textsuperscript{1}, Sho Tsuji\textsuperscript{1}, Page Piccinini\textsuperscript{2}, Molly Lewis\textsuperscript{3}, Mika Braginsky\textsuperscript{3}, Michael C. Frank\textsuperscript{3}, \& Alejandrina Cristia\textsuperscript{1}}

  \def\affdep{{"", "", "", "", "", "", ""}}%
  \def\affcity{{"", "", "", "", "", "", ""}}%

  \affiliation{
    \vspace{0.5cm}
          \textsuperscript{1} Laboratoire de Sciences Cognitives et Psycholinguistique, ENS\\
          \textsuperscript{2} NeuroPsychologie Interventionnelle, ENS\\
          \textsuperscript{3} Department Psychology, Stanford University  }

 % If no author_note is defined give only author information if available
    \authornote{
    \newcounter{author}
                                                                                    }
  
  \note{Correspondence concerning this article should be addressed to Christina
Bergmann, Laboratoire de Sciences Cognitives et Psycholinguistique, ENS.
29 Rue d'Ulm, 75005 Paris, France. E-mail:
\href{mailto:chbergma@gmail.com}{\nolinkurl{chbergma@gmail.com}}}

  \keywords{replicability, reproducibility, meta-analysis, developmental psychology,
language acquisition \\

    \indent Word count: XXXX
  }



  \usepackage{setspace}
  \usepackage{float}
  \usepackage{graphicx}
  \AtBeginEnvironment{tabular}{\singlespacing}
  \usepackage{pbox}

\begin{document}

\maketitle



\section{Introduction}\label{introduction}

Psychology has seen a recent \enquote{crisis of confidence} in key
findings, as many subfields are plagued by issues of low reliability and
validity of their data {[}CITE{]}. Replicability, that is conducting
conceptually similar experiment with new stimuli and in a slightly
different population but following the same procedure and analyses
(based on the published report) with the same outcome as reported
(allowing for a margin of error), is a core concept in this recent
crisis. Being able to (repeatedly) successfully replicate a study can be
taken as an indicator that the phenomenon under investigation is true
and theories can be built on it. This means that a single published
report is not sufficient to establish the existence of a phenomenon, and
misleading reports might be caused by a number of issues. Next to
spurious findings (which can occur even when following best practices),
a number of habits in psychological research might result in outcomes
not reflecting whether or not a phenomenon is present in the population.
These habits include running underpowered studies as well as confining
non-significant results to the file-drawer.

The above mentioned issues are potentially exacerbated in child studies,
because the population under investigation is difficult and costly to
both recruit and test. Small sample sizes and noisy measures are a
consequence, which in turn lead to habitually underpowered studies. It
is thus no surprise that some assume the next \enquote{crisis of
confidence} brought about by low replicability of core effects will be
concerning the field of developmental psychology (Frank).

The present paper aims to quantify both some of the potentially
problematic habits of developmental researchers, and show a way forward.
We are thus adding to a recently emerging literature that critically
examines long-held standards and practices in order to make the whole
field more reliable and robust (Mills-Smith, Spangler, Panneton, \&
Fritz, 2015, Csibra, Hernik, Mascaro, Tatone, \& Lengyel (2016)). To be
able to comment on general trends in the field, we make use of a
collection of meta-analyses on child development. The unique opportunity
afforded by such a dataset, which is harnessing data from thousands of
participants, is that we can quantify patterns important for
experimental practices as well as the role of specific research
questions. More precisely, we can quantify whether across different
topics, current practices differ. Based on such an assessment,
recommendations become possible, that either take the specific of
sub-fields into account or can, if supported by the data, be applied
across the board.

We explain in the next section in detail the meta-analytical approach
and why power is a crucial factor for experimental sciences.

\subsection{The Dataset: MetaLab}\label{the-dataset-metalab}

The subsequent analyses are based on MetaLab, an online collection of
meta-analyses on early language development. Meta-analyses are built on
a collection of standardized effect sizes on a single, well-defined
phenomenon. By accumulating effect sizes and weighting them by their
reliability, it is possible to compute an estimate of the population
effect. Consequently, meta-analyses do not rely on one (possibly false)
study outcome, be it significant or not. Despite their overall utility,
meta-analyses are not frequently conducted in most branches of
developmental psychology. Instead, narrative summaries are the dominant
tool to build theories, and that single studies are cited as evidence
for the presence or absence of an ability instead of meta-analyses.
Currently, MetaLab contains 12 meta-analyses, but it is open to
submissions and updates. The present analyses thus are a snapshot;
through dynamic reports on the website, and by downloading the freely
available data, it is continuously possible to obtain the most recent
data.

In MetaLab, various meta-analyses are combined that address phenomena
ranging from infant-directed speech preference to mutual exclusivity.
Those datasets were either added by the authors (n=XXX) or extracted
from published meta-analyses related to language development (n=2, i.e.
(Colonessi, {\textbf{???}})). In the former case, we attempted to code
as much detail as possible for each entered experiment (note that a
paper can contain many experiments). A high level of detail allows not
only to retrieve general measures of interest, but also to conduct
follow-up analyses into different possible research questions and
prospective power calculations taking as much methodological detail as
possible into account.

Overall, parts of each meta-analysis are standardized to allow for the
computation of common effect size estimates and for analyses that span
different phenomena. These standardized variables include study
descriptors (such as citation and peer review status), participant
characteristics (including mean age and age range, percent female
participants), methodological infomation (for example what dependent
variable was measured), and information necessary to compute effect
sizes (number of participants, if available means and standard
deviations of the dependent measure, otherwise test statistics, such as
t-values or F scores).

As dependent measure, we report Cohen's \emph{d}, a standardized effect
size based on comparing sample means. This effect size was calculated
when possible from sample means and standard deviations across designs
with the appropriate formula. When these data were not available, we
used test statistics, more precisely t-values or F scores of the test
assessing the main hypothesis. We also computed the variance of this
effect size, which allows to weigh each effect size when aggregating
across studies. The variance is mainly determined by the number of
participants; intuitively effect sizes based on larger samples will be
weighted higher. Note that for research designs testing participants in
two conditions that need to be compared (for example exposing the same
infants to infant- and adult-directed speech), correlations between
those two measures are needed to estimate the effect size variance. This
measure is usually not reported. Some correlations could be obtained
through direct contact with the original authors (see e.g., (Bergmann \&
Cristia, 2015) for details), for others we estimated this factor.

Descriptions of all phenomena covered by MetaLab, including which papers
and other sources have been considered, can be found on the companion
website at metalab.stanford.edu and in the supporting information.

\subsection{Average sample size, effect size, and power per
phenomenon}\label{average-sample-size-effect-size-and-power-per-phenomenon}

The table below provides summary information for each meta-analysis in
MetaLab regarding a number of factors, including the number of single
effect sizes and that of papers contributing to a given dataset.
Phenomena differ in the age groups typically tested and the age range
covered. This is of high importance, both theoretically, as younger
infants might generate more noisy behaviors and are not as advanced in
their linguistic abilities, and practically, as older infants might be
subjected to more robust methods and could be a more readily available
participant pool. The typical sample size as well as the minimum and
maximum (allowing to estimate the range in our data) is noted as well.
Based on the meta-analytical effect size and the average number of
participants, we calculated typical power. Note that recommendations are
for this value to be above 80\%, which refers to a likelihood that 4 out
of 5 studies show a significant outcome for an effect truly present in
the population.

Underpowered studies, that is studies with a low probability to detect
an effect given it is present in the population, pose a problem for
branches of developmental studies that interpret both significant and
nonsignificant findings; for example when tracking the emergence of an
ability as children mature or when examining the boundary conditions of
an ability. This practice is problematic for two reasons: On one hand,
the null hypothesis, for example that two groups do not differ, is not
being tested, so it cannot be adopted based on a high p-value. Instead,
p-values can only support rejections of the null hypothesis with a
certainty that the data at hand are incompatible with it below a pre-set
threshold. On the other hand, even in the most rigorous study design and
execution, null results will occur ever so often; for example in a study
with 80\% power (a number typically deemed sufficient), every fifth
result will not reflect that there is a true effect present in the
population. Disentangling whether a non-significant finding indicates
the absence of a skill, random measurement noise, or the lack of
experimental power to detect this skill reliably and with statistical
support is impossible based on p-values.

A second problem emerges when underpowered studies yield significant
outcomes, as the effects reported in such cases will be over-estimating
the true effect. This makes appropriate planning for future research
which aims to build on this report more difficult, as sample sizes will
be too small, leading to null-results which do not speak to the
phenomenon under investigation. This poses a serious hindrance for work
building on seminal studies, including replications across languages and
extensions. However, aggregating over such null-results using a graded
estimate, i.e.~a standardized effect size, can reveal whether a
phenomenon is present in the population and correct for the initial
over-estimation. In short, even a true positive result is insufficient
in the quest for the truth when it is underpowered.

\begin{longtable}[c]{@{}lrrrrrrrrr@{}}
\caption{Descriptions of meta-analyses currently in
MetaLab.}\tabularnewline
\toprule
Meta Analysis (MA) & Mean Age in Months & Mean Sample Size & Min. Sample
Size & Max. Sample Size & \# Effect Sizes & \# Papers & d & SE & Avg
Power\tabularnewline
\midrule
\endfirsthead
\toprule
Meta Analysis (MA) & Mean Age in Months & Mean Sample Size & Min. Sample
Size & Max. Sample Size & \# Effect Sizes & \# Papers & d & SE & Avg
Power\tabularnewline
\midrule
\endhead
Gaze following & 13.63 & 31.61 & 12 & 63 & 33 & 11 & 1.14 & 0.18 &
0.99\tabularnewline
Infant directed speech preference & 4.72 & 22.11 & 10 & 60 & 50 & 16 &
0.77 & 0.15 & 0.70\tabularnewline
Label advantage in concept learning & 10.96 & 16.44 & 9 & 32 & 100 & 17
& 0.35 & 0.06 & 0.16\tabularnewline
Mutual exclusivity & 27.68 & 18.83 & 8 & 72 & 60 & 19 & 0.83 & 0.14 &
0.70\tabularnewline
Online word recognition & 20.30 & 39.40 & 16 & 95 & 15 & 6 & 1.18 & 0.22
& 1.00\tabularnewline
Phonotactic learning & 10.18 & 19.45 & 8 & 40 & 47 & 15 & 0.12 & 0.07 &
0.07\tabularnewline
Pointing and vocabulary (concurrent) & 21.03 & 26.58 & 6 & 50 & 12 & 12
& 0.98 & 0.18 & 0.94\tabularnewline
Pointing and vocabulary (longitudinal) & 18.51 & 32.22 & 12 & 72 & 18 &
18 & 0.54 & 0.12 & 0.56\tabularnewline
Sound symbolism & 11.76 & 19.02 & 11 & 26 & 42 & 10 & 0.25 & 0.11 &
0.12\tabularnewline
Statistical sound category learning & 7.46 & 16.35 & 5 & 35 & 18 & 9 &
-0.29 & 0.15 & 0.13\tabularnewline
Vowel discrimination (native) & 7.51 & 16.00 & 6 & 50 & 143 & 32 & 0.62
& 0.09 & 0.40\tabularnewline
Vowel discrimination (non-native) & 8.08 & 17.69 & 8 & 30 & 48 & 15 &
1.00 & 0.30 & 0.82\tabularnewline
Word segmentation & 9.20 & 22.14 & 4 & 64 & 291 & 66 & 0.17 & 0.03 &
0.08\tabularnewline
\bottomrule
\end{longtable}

\subsubsection{Power: Comparing meta-analytic effect size and oldest
paper}\label{power-comparing-meta-analytic-effect-size-and-oldest-paper}

As Table 1 shows, experimenters are habitually not including a
sufficient number of participants to observe a given effect, assuming
the meta-analytic estimate for a given topic. It might, however, be
possible, that power has been determined based on a seminal paper to be
replicated. Initial reports tend to overestimate effect sizes
{[}CITE{]}, possibly explaining the lack of power in some sub-domains.
We extracted for each dataset the oldest paper and therein the largest
reported effect size and re-calculated power accordingly. The results
are shown in the table below. It turns out that in some cases, such as
native and non-native vowel discrimination, sample size choices match
well with the oldest report.

\begin{longtable}[c]{@{}llrrr@{}}
\caption{For each meta-analysis, largest d from oldest paper and
meta-analytic d.}\tabularnewline
\toprule
Meta-analysis (MA) & Oldest Paper & Oldest d & Mean Sample Size &
Power\tabularnewline
\midrule
\endfirsthead
\toprule
Meta-analysis (MA) & Oldest Paper & Oldest d & Mean Sample Size &
Power\tabularnewline
\midrule
\endhead
Gaze following & Mundy \& Gomes (1998) & 4.52 & 31.61 &
1.00\tabularnewline
Infant directed speech preference & Glenn \& Cunningham (1983) & 2.56 &
22.11 & 1.00\tabularnewline
Label advantage in concept learning & Balaban \& Waxman (1997) & 0.86 &
16.44 & 0.67\tabularnewline
Mutual exclusivity & Merriman et al. (1989) & 0.70 & 18.83 &
0.55\tabularnewline
Online word recognition & Zangl et al. (2005) & 0.89 & 39.40 &
0.97\tabularnewline
Phonotactic learning & Chambers et al. (2003) & 0.98 & 19.45 &
0.84\tabularnewline
Pointing and vocabulary (concurrent) & Murphy (1978) & 0.65 & 26.58 &
0.65\tabularnewline
Pointing and vocabulary (longitudinal) & Bates et al. (1979) & 0.56 &
32.22 & 0.60\tabularnewline
Sound symbolism & Maurer, Pathman, \& Mondloch (2006) & 0.95 & 19.02 &
0.82\tabularnewline
Statistical sound category learning & Maye, Werker, \& Gerken (2002) &
0.56 & 16.35 & 0.34\tabularnewline
Vowel discrimination (native) & Trehub (1973) & 1.87 & 16.00 &
1.00\tabularnewline
Vowel discrimination (non-native) & Trehub (1976) & 1.02 & 17.69 &
0.84\tabularnewline
Word segmentation & Jusczyk \& Aslin (1995) & 0.56 & 22.14 &
0.44\tabularnewline
\bottomrule
\end{longtable}

\begin{figure}[htbp]
\centering
\includegraphics{EducationPaper_files/figure-latex/Plot of Difference of d Values-1.pdf}
\caption{Correlation of largest d from oldest paper and difference
between oldest d and meta-analytic d.}
\end{figure}

To illustrate the disparity between the oldest effect size and the
meta-analytic effect, we plot the difference between both against the
oldest effect. This difference is larger as oldest effect size
increases, with an average of 0.65 compared with an average effect size
of 0.63 (note that we based this on the absolute value). The plot
showcases that researchers might want to be wary of large effects, as
they are more likely to be non-representative of the true phenomenon
compared to smaller initial effects being reported. Especially when
making decisions about sample sizes, large effect might thus not be the
best guide. Taking the above-mentioned mean values as example, a
realistic sample size to ensure 80\% power would be 40.15 participants,
instead of 10.59 participants suggested by the first paper. While these
numbers average over research questions and methods, which all influence
the specific number of participants necessary, this example showcases
that experimenters should take into account as much evidence as
available to be able tp plan for robust and reproducible studies.

\subsection{What is the effect of method
choice?}\label{what-is-the-effect-of-method-choice}

The number of paradigms available in developmental research when testing
infants and children, is somewhat limited by a number of factors, such
as time. Nonetheless, often there is more than one way to measure a
specific construct available. Consider a measurement of preference, for
example when trying to establish that infants distinguish
infant-directed from adult-directed speech and in fact prefer the
former. This preference can be measured in a number of ways, as it is
something children bring to the lab. In the meta-analysis on IDS
preference there are 4 different methods, all aiming to pick up the very
same phenomenon, and with this approach this specific line of
investigation is no exception, as 4 datasets of the 13 included datasets
contain 3 or more methods.

Choosing a robust method can also help increase the power of studies,
such that more precise measurements lead to larger effects and thus
require fewer participants to be tested. However, the number of
participants relates to the final sample and not how many infants had to
be invited into the lab. We thus first quantify whether methods differ
in their typical drop-out rate. To this end we consider all methods
across datasets in metalab which have more than 10 records. The results
of the linear mixed effect model predicting droout rate by method and
mean participant age (with dataset as fixed factor to account for the
different effects being tested) are summarized in the table below. The
results show that, taking central fixation as baseline, conditioned
headturn and stimulus alternation have significantly more drop-outs.
These effects seem to hold across age groups.

\begin{longtable}[c]{@{}lrrr@{}}
\caption{Method vs Dropout}\tabularnewline
\toprule
& Estimate & Std. Error & t value\tabularnewline
\midrule
\endfirsthead
\toprule
& Estimate & Std. Error & t value\tabularnewline
\midrule
\endhead
(Intercept) & 0.2830431 & 0.0247088 & 11.4551620\tabularnewline
methodconditioned head-turn & 0.2693826 & 0.0541791 &
4.9720801\tabularnewline
methodhead-turn preference procedure & -0.0249028 & 0.0337799 &
-0.7372074\tabularnewline
methodstimulus alternation & 0.2161717 & 0.0416199 &
5.1939453\tabularnewline
ageC & 0.0026638 & 0.0046998 & 0.5667801\tabularnewline
methodconditioned head-turn:ageC & 0.0214482 & 0.0114067 &
1.8803223\tabularnewline
methodhead-turn preference procedure:ageC & 0.0122659 & 0.0071521 &
1.7150021\tabularnewline
methodstimulus alternation:ageC & -0.0012706 & 0.0093964 &
-0.1352257\tabularnewline
\bottomrule
\end{longtable}

\begin{figure}[htbp]
\centering
\includegraphics{EducationPaper_files/figure-latex/Plot of Effect of Method on Dropout Rate-1.pdf}
\caption{Percent dropout as explained by different methods and mean age
of participants.}
\end{figure}

\begin{longtable}[c]{@{}lrrrrrr@{}}
\caption{Effect of d by method with central fixation as baseline
method.}\tabularnewline
\toprule
& estimate & se & zval & pval & ci.lb & ci.ub\tabularnewline
\midrule
\endfirsthead
\toprule
& estimate & se & zval & pval & ci.lb & ci.ub\tabularnewline
\midrule
\endhead
intrcpt & 0.5132018 & 0.1156527 & 4.4374395 & 0.0000091 & 0.2865267 &
0.7398769\tabularnewline
ageC & 0.0140214 & 0.0068290 & 2.0532200 & 0.0400512 & 0.0006368 &
0.0274059\tabularnewline
relevel(method, \enquote{central fixation})anticipatory eye movements &
-1.3797138 & 9.8476930 & -0.1401053 & 0.8885768 & -20.6808373 &
17.9214098\tabularnewline
relevel(method, \enquote{central fixation})conditioned head-turn &
1.6519226 & 0.3462236 & 4.7712590 & 0.0000018 & 0.9733368 &
2.3305085\tabularnewline
relevel(method, \enquote{central fixation})forced-choice & 1.0215165 &
0.3167291 & 3.2252061 & 0.0012588 & 0.4007389 & 1.6422940\tabularnewline
relevel(method, \enquote{central fixation})head-turn preference
procedure & -0.0250973 & 0.2478096 & -0.1012765 & 0.9193310 & -0.5107951
& 0.4606005\tabularnewline
relevel(method, \enquote{central fixation})high-amplitude sucking &
-1.2312867 & 4.1786021 & -0.2946647 & 0.7682500 & -9.4211963 &
6.9586229\tabularnewline
relevel(method, \enquote{central fixation})hybrid visual habituation
procedure & -4.7403364 & 2.8763087 & -1.6480625 & 0.0993399 &
-10.3777979 & 0.8971251\tabularnewline
relevel(method, \enquote{central fixation})looking while listening &
-0.3061340 & 0.2380212 & -1.2861626 & 0.1983863 & -0.7726471 &
0.1603790\tabularnewline
relevel(method, \enquote{central fixation})oddball & -0.3026225 &
0.4108786 & -0.7365255 & 0.4614110 & -1.1079298 &
0.5026847\tabularnewline
relevel(method, \enquote{central fixation})stimulus alternation &
-0.0496753 & 0.2256594 & -0.2201341 & 0.8257667 & -0.4919597 &
0.3926090\tabularnewline
relevel(method, \enquote{central fixation})word-object pairing &
-0.3321379 & 0.5057796 & -0.6566851 & 0.5113834 & -1.3234476 &
0.6591718\tabularnewline
ageC:relevel(method, \enquote{central fixation})anticipatory eye
movements & -0.4723527 & 3.1999856 & -0.1476109 & 0.8826499 & -6.7442092
& 5.7995038\tabularnewline
ageC:relevel(method, \enquote{central fixation})conditioned head-turn &
0.1468464 & 0.0592087 & 2.4801498 & 0.0131327 & 0.0307995 &
0.2628933\tabularnewline
ageC:relevel(method, \enquote{central fixation})forced-choice &
-0.0392280 & 0.0211849 & -1.8516968 & 0.0640694 & -0.0807496 &
0.0022936\tabularnewline
ageC:relevel(method, \enquote{central fixation})head-turn preference
procedure & -0.0275691 & 0.0270572 & -1.0189186 & 0.3082416 & -0.0806003
& 0.0254621\tabularnewline
ageC:relevel(method, \enquote{central fixation})high-amplitude sucking &
-0.2315320 & 0.4806216 & -0.4817344 & 0.6299946 & -1.1735330 &
0.7104691\tabularnewline
ageC:relevel(method, \enquote{central fixation})hybrid visual
habituation procedure & -0.8201068 & 0.6222136 & -1.3180470 & 0.1874879
& -2.0396230 & 0.3994094\tabularnewline
ageC:relevel(method, \enquote{central fixation})looking while listening
& -0.0090959 & 0.0145594 & -0.6247419 & 0.5321405 & -0.0376317 &
0.0194400\tabularnewline
ageC:relevel(method, \enquote{central fixation})oddball & -0.0643442 &
0.0521788 & -1.2331484 & 0.2175204 & -0.1666129 &
0.0379244\tabularnewline
ageC:relevel(method, \enquote{central fixation})stimulus alternation &
0.0007789 & 0.0288999 & 0.0269516 & 0.9784983 & -0.0558639 &
0.0574217\tabularnewline
\bottomrule
\end{longtable}

We built a meta-analytic model with the effect size measure Cohen's
\emph{d} as the dependent variable, method and mean age of population
centered as independent variables. The model also includes the variance
of \emph{d} for sampling variance, and paper within meta-analysis as a
random effect (because we assume that within a paper experiments and
thus effect sizes will be more similar to each other than across
papers). Only methods with at least 20 associated effect sizes in
MetaLab were included in the model. Thus, the present analyses are
limited to 283 observations. Since the model compares one method as the
baseline to all other methods, a baseline method had to be chosen.
\enquote{Central fixation} was included as the baseline method, as it
appears most frequently in the 4 datasets included in this analysis (107
times out of 283 total entries of the selected meta-analyses).

\begin{figure}[htbp]
\centering
\includegraphics{EducationPaper_files/figure-latex/Plot of Effect of Method-1.pdf}
\caption{Effect size as explained by different methods and mean age of
participants.}
\end{figure}

\textbf{\emph{TO DO: Add caveats}}

\section{General Discussion}\label{general-discussion}

\subsection{Recommendation: appreciate meta-analyses
more}\label{recommendation-appreciate-meta-analyses-more}

Meta-analyses and meta-analytic thinking might help solve some of the
issues we uncovered.

The reluctance to appreciate meta-analyses is evident when comparing
citation rates of the initial paper with a meta-analysis on the same
phenomenon. Consider the example of infant-directed speech preference,
where infants listen longer to speech stimuli showing the typical
characteristics of parents talking to their young children. This
phenomenon is both theoretically and practically highly relevant and
thus receives substantial attention from the field, not the least in a
recent large-scale replication attempt (Frank). A meta-analysis on this
phenomenon was published in 2012 ({\textbf{???}}), taking 34 studies
into account. The oldest paper stems from 1983 {[}Cite Glenn \&
Cunningham \enquote{What do babies listen to most? A developmental study
of auditory preferences in nonhandicapped infants and infants with
Down's syndrome.}{]}, and the seminal work (measured by the number of
citations) was published in 1990 {[}CITE Cooper Aslin
\enquote{Preference for infant-directed speech in the first month after
birth}{]}. Comparing these three papers by the number of citations
divided by the years since publication (retreived from google scholar on
September 2, 2016) shows that the seminal paper is cited an order of
magnitude more every year (on average 24.3 times) than the meta-analysis
(2.75 times). This is indicative of practices both when constructing
theories and planning experiment: The quantified evidence is
under-appreciated, despite providing a number of useful measures, such
as effect sizes for different age groups and for various methodological
decisions such as stimulus type (synthetic versus natural speech, the
own mother versus a stranger, among other things). This is both highly
relevant for theories, as the observation of an increased preference for
infant-directed speech is a qualitative observation that can allow for
more fine-grained hypothesizing. Practically, the information about
effect size changes and the impact of method allow for more robust
experiment planning and power calculations. We will come back to the
issue of power and the impact of considering a seminal paper versus a
meta-analysis below. Similar observations hold for other meta-analyses
currently available {[}CITE inphondb, inworddb, others outside language
development?{]}.

While anecdotal, this survey showcases current practices and points to
one reason for underpowered studies. If authors only consider a single
seminal paper to estimate the number of participants necessary, they
might habitually run under-powered studies. We show this in dedicated
analyses on meta-analytic versus seminal effect size and the resulting
typical power in a literature.

\subsubsection{Why don't we do MAs?}\label{why-dont-we-do-mas}

Meta-analyses are also seldomly conducted. This is due to high hurdles
and few rewards. Conducting a meta-analysis is a laborious process,
particularly according to common practice where only a few people do the
work, with little ready-to-use support tools and educational materials
available. Incentives for creating meta-analyses are low, as public
recognition is tied to a single publication. The benefits of
meta-analyses for the field, for instance the possibility to conduct
power analyses, are often neither evident nor accessible to individual
researchers, as the data are not shared and traditional meta-analyses
remain static after publication, aging quickly as new results emerge.

A final impediment to meta-analyses in developmental science are, as we
illustrated in more detail in section XXX, current reporting standards,
which make it difficult and at times even impossible to compute effect
sizes from the published literature. As consequence, both systematic,
full-scale meta-analyses, and a targeted priori calculation of power and
thus the determination of appropriate sample sizes are not yet common
practice. Our analyses span various journals and publication years and
are thus complementary to recent reports on the overall lack of power in
(developmental) psychology based on single-journal/-year samples (e.g.,
Marszalek, Barber, Kohlhart, \& Holmes, 2011).

\subsection{Going forward: How can MetaLab help change
practices?}\label{going-forward-how-can-metalab-help-change-practices}

MetaLab is built on two core principles: lowering hurdles to foster the
implementation of practices which are rapidly becoming standard
procedure, not only in other branches of psychology (such as effect size
estimation and power calculation), and crowdsourcing to decrease the
workload of single researchers. MetaLab is based on the recently
proposed concept of community-augmented meta-analyses (CAMAs; Tsuji,
Bergmann, \& Cristia, 2014), which combine meta-analyses and open
repositories. The advantages of this union are that meta-analyses are
shared and get updated continuously, so they can capture the most recent
state of the literature and are open to contributions of unpublished
results.

MetaLab expands on CAMAs by providing an infrastructure for a range of
uses. It is possible to gain an overview of the literature, get insights
into specific topics through dynamically rendered reports, conduct power
calculations, and contribute not only single recent or unpublished
datasets, but whole meta-analyses that can then be opened to
contributions and analysis. All meta-analyses share a core of 20
variables which not only allow for the computation of effect sizes
across vastly different studies, but also provide the basis for further
comparisons. These comparisons are both of practical and theoretical
importance, for example can we compare which method is more robust and
suitable for various ages. Researchers then can both better compare
existent findings and plan their own research to be more effective. This
becomes possible by focusing on a high-level but specific and
constrained topic, in the case of MetaLab this is early language
development and adjacent phenomena.

MetaLab also poses several advantages compared to existing software for
meta-analyses. First, adding meta-analyses is supported not only by
sharing standardized formats but also by offering guidance in
identifying the correct data to enter, based on the extant data and
examples from the developmental psychology literature. Further, novice
users can easily engage with the platform to estimate, for example,
effect sizes, or decide on sample sizes with a simple interactive tool.
Secondly, since all data and scripts are freely and openly available, it
is possible to inspect and if needed correct all computations. Errors
are thus removed much more swiftly than would be the case for
(commercial) software, without losing the benefit of a stable platform.
By changing current practices, we aim to increase the reliability of
developmental findings and thus the credibility of the field, which has
recently come under fire (e.g., Peterson, 2016).

\subsection{With these tools, what do we have to
do?}\label{with-these-tools-what-do-we-have-to-do}

{[}Tutorial section{]}

On the individual level:

\begin{itemize}
\itemsep1pt\parskip0pt\parsep0pt
\item
  How to determine participants: Power calculator, typical N in the
  field
\item
  How to run the best possible study: Make design choices to have a more
  robust measure (smaller sample and more power)
\item
  How do I report my data? Best reporting practices (include
  correlations for within, always report means and SD); and possibly
  best visualization practices
\end{itemize}

Further individual benefits:

\begin{itemize}
\itemsep1pt\parskip0pt\parsep0pt
\item
  Don't despair when a null result occurs, you can still help the
  community with it
\item
  For replication / training purposes possible to compare ES and select
  robust ones
\end{itemize}

On the general level:

\begin{itemize}
\itemsep1pt\parskip0pt\parsep0pt
\item
  Evidence becomes more reliable
\item
  New evidence can be integrated with previous work directly without
  much effort
\item
  Complete, unbiased overview of a research literature

  \begin{itemize}
  \itemsep1pt\parskip0pt\parsep0pt
  \item
    Identify unexplained variance
  \item
    Where are gaps?
  \item
    Which moderators (do not) affect outcomes

    \begin{itemize}
    \itemsep1pt\parskip0pt\parsep0pt
    \item
      Examples from published MAs:

      \begin{itemize}
      \itemsep1pt\parskip0pt\parsep0pt
      \item
        InWordDB lack of age effect (predicted and strongly assumed in
        the field)
      \item
        InPhonDB confirmation of diverging effects for native /
        nonnative, with a quantitative timeline
      \end{itemize}
    \end{itemize}
  \end{itemize}
\end{itemize}

\section*{References}\label{references}
\addcontentsline{toc}{section}{References}

Bergmann, C., \& Cristia, A. (2015). Development of infants'
segmentation of words from native speech: A meta-analytic approach.
\emph{Developmental Science}.

Csibra, G., Hernik, M., Mascaro, O., Tatone, D., \& Lengyel, M. (2016).
Statistical treatment of looking-time data. \emph{Developmental
Psychology}, \emph{52}(4), 521--536.

Frank, M. C. (). A collaborative approach to infant research: Promoting
reproducibility, best practices, and theory-building.

Marszalek, J. M., Barber, C., Kohlhart, J., \& Holmes, C. B. (2011).
Sample size in psychological research over the past 30 years 1, 2.
\emph{Perceptual and Motor Skills}, \emph{112}(2), 331--348.

Mills-Smith, L., Spangler, D. P., Panneton, R., \& Fritz, M. S. (2015).
A missed opportunity for clarity: Problems in the reporting of effect
size estimates in infant developmental science. \emph{Infancy},
\emph{20}(4), 416--432.

Peterson, D. (2016). The baby factory: Difficult research objects,
disciplinary standards, and the production of statistical significance.
\emph{Socius: Sociological Research for a Dynamic World}, \emph{2},
1--10.

Tsuji, S., Bergmann, C., \& Cristia, A. (2014). Community-augmented
meta-analyses: Toward cumulative data assessment. \emph{Psychological
Science}, \emph{9}(6), 661--665.





\end{document}
