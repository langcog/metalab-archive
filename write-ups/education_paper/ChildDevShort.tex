\documentclass[english,floatsintext,man]{apa6}

\usepackage{amssymb,amsmath}
\usepackage{ifxetex,ifluatex}
\usepackage{fixltx2e} % provides \textsubscript
\ifnum 0\ifxetex 1\fi\ifluatex 1\fi=0 % if pdftex
  \usepackage[T1]{fontenc}
  \usepackage[utf8]{inputenc}
\else % if luatex or xelatex
  \ifxetex
    \usepackage{mathspec}
    \usepackage{xltxtra,xunicode}
  \else
    \usepackage{fontspec}
  \fi
  \defaultfontfeatures{Mapping=tex-text,Scale=MatchLowercase}
  \newcommand{\euro}{€}
\fi
% use upquote if available, for straight quotes in verbatim environments
\IfFileExists{upquote.sty}{\usepackage{upquote}}{}
% use microtype if available
\IfFileExists{microtype.sty}{\usepackage{microtype}}{}

% Table formatting
\usepackage{longtable, booktabs}
\usepackage{lscape}
% \usepackage[counterclockwise]{rotating}   % Landscape page setup for large tables
\usepackage{multirow}		% Table styling
\usepackage{tabularx}		% Control Column width
\usepackage[flushleft]{threeparttable}	% Allows for three part tables with a specified notes section
\usepackage{threeparttablex}            % Lets threeparttable work with longtable

% Create new environments so endfloat can handle them
% \newenvironment{ltable}
%   {\begin{landscape}\begin{center}\begin{threeparttable}}
%   {\end{threeparttable}\end{center}\end{landscape}}

\newenvironment{lltable}
  {\begin{landscape}\begin{center}\begin{ThreePartTable}}
  {\end{ThreePartTable}\end{center}\end{landscape}}




% The following enables adjusting longtable caption width to table width
% Solution found at http://golatex.de/longtable-mit-caption-so-breit-wie-die-tabelle-t15767.html
\makeatletter
\newcommand\LastLTentrywidth{1em}
\newlength\longtablewidth
\setlength{\longtablewidth}{1in}
\newcommand\getlongtablewidth{%
 \begingroup
  \ifcsname LT@\roman{LT@tables}\endcsname
  \global\longtablewidth=0pt
  \renewcommand\LT@entry[2]{\global\advance\longtablewidth by ##2\relax\gdef\LastLTentrywidth{##2}}%
  \@nameuse{LT@\roman{LT@tables}}%
  \fi
\endgroup}


  \usepackage{graphicx}
  \makeatletter
  \def\maxwidth{\ifdim\Gin@nat@width>\linewidth\linewidth\else\Gin@nat@width\fi}
  \def\maxheight{\ifdim\Gin@nat@height>\textheight\textheight\else\Gin@nat@height\fi}
  \makeatother
  % Scale images if necessary, so that they will not overflow the page
  % margins by default, and it is still possible to overwrite the defaults
  % using explicit options in \includegraphics[width, height, ...]{}
  \setkeys{Gin}{width=\maxwidth,height=\maxheight,keepaspectratio}
\ifxetex
  \usepackage[setpagesize=false, % page size defined by xetex
              unicode=false, % unicode breaks when used with xetex
              xetex]{hyperref}
\else
  \usepackage[unicode=true]{hyperref}
\fi
\hypersetup{breaklinks=true,
            pdfauthor={},
            pdftitle={Assessing methodological practices in language acquisition research through meta-analyses},
            colorlinks=true,
            citecolor=blue,
            urlcolor=blue,
            linkcolor=black,
            pdfborder={0 0 0}}
\urlstyle{same}  % don't use monospace font for urls

\setlength{\parindent}{0pt}
%\setlength{\parskip}{0pt plus 0pt minus 0pt}

\setlength{\emergencystretch}{3em}  % prevent overfull lines

\ifxetex
  \usepackage{polyglossia}
  \setmainlanguage{}
\else
  \usepackage[english]{babel}
\fi

% Manuscript styling
\captionsetup{font=singlespacing,justification=justified}
\usepackage{csquotes}
\usepackage{upgreek}

 % Line numbering
  \usepackage{lineno}
  \linenumbers


\usepackage{tikz} % Variable definition to generate author note

% fix for \tightlist problem in pandoc 1.14
\providecommand{\tightlist}{%
  \setlength{\itemsep}{0pt}\setlength{\parskip}{0pt}}

% Essential manuscript parts
  \title{Assessing methodological practices in language acquisition research
through meta-analyses}

  \shorttitle{Methodological practices in language acquisition research}


  \author{Christina Bergmann\textsuperscript{1}, Sho Tsuji\textsuperscript{2}, Page E. Piccinini\textsuperscript{3}, Molly L. Lewis\textsuperscript{4}, Mika Braginsky\textsuperscript{5}, Michael C. Frank\textsuperscript{6}, \& Alejandrina Cristia\textsuperscript{1}}

  \def\affdep{{"", "", "", "", "", "", ""}}%
  \def\affcity{{"", "", "", "", "", "", ""}}%

  \affiliation{
    \vspace{0.5cm}
          \textsuperscript{1} Ecole Normale Sup\{'e\}rieure, PSL Research University, D\{'e\}partement
d'Etudes Cognitives, Laboratoire de Sciences Cognitives et
Psycholinguistique (ENS, EHESS, CNRS)\\
          \textsuperscript{2} University of Pennsylvania, Department of Psychology\\
          \textsuperscript{3} Ecole Normale Sup\{'e\}rieure, PSL Research University, D\{'e\}partement
d'Etudes Cognitives, Neuropsychologie Interventionnelle (ENS, EHESS,
CNRS)\\
          \textsuperscript{4} University of Chicago, Computation Institute/University of
Wisconsin-Madison, Department of Psychology\\
          \textsuperscript{5} Massachusetts Institute of Technology, Department of Brain and Cognitive
Sciences\\
          \textsuperscript{6} Stanford University, Department of Psychology, Language and Cognition
Lab  }

 % If no author_note is defined give only author information if available
    \authornote{
    \newcounter{author}
                      Correspondence concerning this article should be addressed to Christina Bergmann, Ecole Normale Su\{'e\}rieure, Laboratoire de Sciences Cognitives et
Psycholinguistique, 29, rue d'Ulm, 75005 Paris, France.. E-mail: \href{mailto:chbergma@gmail.com}{\nolinkurl{chbergma@gmail.com}}
                                                                            }
  

  \abstract{Replicability is a critical feature of scientific research. Recent
concerns about replicability in psychology have led to a focus on
statistical power, the probability of experiments to detect particular
effects. Using data from a new database of meta-analyses, we analyze
methodological trends in the language development literature. Although
statistical power has been a concern in infancy research, no extant data
speak to the average level of power in this area. We calculated the
typical statistical power for experiments across our database by
comparing sample sizes in each experiment to the meta-analytic estimate
of the effect size. With a median effect size of Cohen's \emph{d} = .57
across all 12 phenomena, and a typical sample size of 17 participants
per cell, power is at 60\%. This finding suggests that typical sample
sizes in infancy research are likely too low and that researchers do not
habitually consider effect sizes in their experiment planning. We also
show that seminal publications in the early language learning literature
typically over-estimate effect sizes relative to later investigations,
but that this literature does not show evidence of ``p-hacking''
(undisclosed analytic flexibility). We conclude with recommendations for
experimental planning and reporting as well as for the use of
meta-analysis in developmental research.}
  \keywords{replicability, reproducibility, meta-analysis, language acquisition,
power \\

    \indent Word count: X
  }





\begin{document}

\maketitle

\setcounter{secnumdepth}{0}



Empirical research is built on a never-ending conversation between
theory and data, between expectations and observations. Theories lead to
new experimental questions and new data in turn help us refine our
theories. This process relies crucially on access to reliable empirical
data. Unfortunately, investigators of the scientific process have noted
that the assessment of the value of empirical data points can be biased
by concerns about publishability (Nosek, Spies, \& Motyl, 2012), which
in turn often depends on the observation of statistically significant
and theoretically-surprising outcomes. If researchers aim for
publishability, this is likely to lead to practices that undermine the
quality and reliability of their data. It has therefore been suggested
that theories should rely on replicable findings. Replicability is
crucial in experimental sciences, particularly for developmental
research: Theories should be based on robust findings and their boundary
conditions have to be explored with sufficiently powered studies to
avoid an excess of false negatives. Further, translating findings on
child development into practice requires a solid knowledge base.

According to some, inappropriate research and reporting practices may be
to blame for the surprisingly high proportion of non-replicable findings
in psychology (Simmons, Nelson, \& Simonsohn, 2011). Simulating the
scientific process, Ioannidis (2005) speculated that most empirical
research findings may even be false. The proportion of false findings in
these simulations was dependent on several features, including the
underlying effect size of a particular phenomenon, the typical sample
sizes used by researchers, and the degree of flexibility in data
collection and analysis. All of these factors are highly relevant to
developmental research.

In the current paper, we survey and quantify methodological practices in
developmental research using meta-analytic tools, focusing on language
development. We take a different approach from the typical meta-analysis
by aggregating over multiple datasets. Using a collection of
standardized meta-analyses, we focus on key experimental design choices:
sample size (and the ensuing statistical power) and experimental method.
In doing so, we provide what is, to our knowledge, the first assessment
of typical practices of developmental research. Based on our findings
and experiences with building meta-analyses and using meta-analytic
tools, we end this paper with suggestions for change.

The data we analyze are part of MetaLab, a database of meta-analyses of
language acquisition that, covers a variety of methods (11 in total) and
participant ages, from newborns to 3.50-year-olds. Since our approach is
accompanied by extensive educational materials, completely open data and
scripts, and we build on open source software (particularly R; R Core
Team, 2016), our approach can easily be extended to other domains of
child development research and we strongly encourage fellow researchers
to build similar collections of meta-analyses describing and quantifying
phenomena in their sub-domain of developmental research.

\subsection{Key concerns for robust research in developmental
science}\label{key-concerns-for-robust-research-in-developmental-science}

In this section we review potential hindrances to developmental research
being robust and reproducible, and briefly describe how we will assess
the status quo. Note that all these descriptions are by necessity brief,
for extended discussions we provide references to suitable readings.

\subsubsection{Statistical power}\label{statistical-power}

Power refers to the probability of detecting an effect and correctly
rejecting the null hypothesis if an effect is indeed present in a
population; power is therefore dependent on the underlying effect size
and the sample size. Of course, low power is problematic in terms of
increased chances of type-II errors (i.e., failure to find a significant
result when there is an underlying effect). It has become increasingly
clear that low power is also problematic in the case of type-I errors,
or false positives, as the effects reported in such cases will be
over-estimating the true effect (Button et al., 2013; see also
Ioannidis, 2005; Simmons et al., 2011). This makes appropriate planning
for future research more difficult, as sample sizes will be too small,
leading to null results due to insensitive research designs rather than
the absence of the underlying effect. This poses a serious hindrance for
work building on seminal studies, including replications and extensions.

Underpowered studies pose an additional and very serious problem for
developmental researchers that interpret significant findings as
indicating that a skill is \enquote{present} and non-significant
findings as a sign that it is \enquote{absent}. In fact, even in the
most rigorous study design and execution, null results will occur
regularly; consider a series of studies with 80\% power (a number
typically deemed sufficient), where every fifth result will be a false
negative, that means it will not reflect that there is a true effect
present in the population. This observation was recently demonstrated by
({\textbf{???}}) by using data from a high-powered looking time study.

To investigate the status quo, we first compute typical power per
phenomenon, based on meta-analytic effect sizes and typical sample size.
We explore which effect sizes would be detectable with the sample sizes
present in our datasets. We additionally investigate how researchers
might determine sample sizes using a different heuristic, following the
first paper on their phenomenon of interest.

\subsubsection{Method choice}\label{method-choice}

Improving procedures in developmental research can be considered both an
economical and ethical necessity, because the population is difficult to
recruit and test. For this reason, developmentalists often
\enquote{tweak} paradigms and develop new ones to increase reliability
and robustness, all with the aim of obtaining a clearer signal.
Especially given the time constraints, we aim to collect a maximum of
data in the short time span infants and children are willing to
participate in a study. Emerging technologies, such as eye-tracking and
tablets, have consequently been eagerly adopted (Frank, Sugarman,
Horowitz, Lewis, and Yurovsky, 2016). As a result, multiple ways to tap
into the same phenomenon exist; consider for example the fact that both
headturn-based paradigms and offline as well as online measurements of
eye movements are frequently being employed to measure infant-directed
speech preference (Dunst, Gorman, \& Hamby, 2012; ManyBabies
Collaborative, 2017).

It remains an open question to what extent these different methods lead
to comparable results. It is possible that some are more robust, but it
is difficult to extract such information based on single studies that
use different materials and test various age groups (but see the
large-scale experimental approach by ManyBabies Collaborative, 2017).
Aggregating over experimental results via meta-analytic tools, in
contrast, allows us to extract general patterns of higher or lower noise
by comparison of effect sizes, which are directly affected by the
variance of the measurement.

We will assess in how far the different methods used in the present
collection of meta-analyses vary in the resulting effect size. Further,
taking possible resource limitations into account, we consider drop-out
rates as a potential measure of interest and discuss whether higher
exclusion rates coincide with more precise measures, yielding higher
effect sizes.

\subsubsection{Questionable research
practices}\label{questionable-research-practices}

Undisclosed flexibility during data collection and analysis is a problem
independent of the availability of various methods to conduct
developmental studies. To begin with, using flexible stopping rules,
where the decision to stop or continue testing depends on the result of
a statistical test, increases the likelihood to obtain a
\enquote{significant} outcome well beyond the traditional 5\%.

As for analytic flexibility, researchers might conduct multiple
significance tests with several more or less related dependent variables
without correcting for this practice. In developmental research, this
encompasses transforming the same measured data into multiple dependent
variables (such as mean scores, difference scores, percentages, and so
on) as well as selectively excluding trials and re-testing the new data
for statistical significance. Next, multiple conditions that selectively
can be dropped from the final report increase the number of significance
tests. Finally, it is problematic to post hoc introduce covariates, most
prominently gender, and test for an interaction with the main effect,
and solely report those outcomes as confirmatory hypothesis test.
Combining two or more of these strategies again increase the number of
significant results that occur by chance even if there is no effect
present in the population. All these practices might seem innocuous and
geared towards \enquote{bringing out} an effect the researcher believes
is real, yet they can inflate the number of significant \emph{p} values,
effectively rendering \emph{p} values and the notion of statistical
significance meaningless (Ioannidis, 2005; Simmons et al., 2011).

It is typically not possible to assess whether flexibility led to a
false positive in a given report. However, we can measure
\enquote{symptoms} of such practices in a whole literature. We focus in
this paper on flexibility in stopping, a practice that was found to be
present, but not predominant in infancy research in a recent anonymous
survey (Eason, Hamlin, \& Sommerville, 2017). Since our data span over
-44 years (publications date range from 1973 to 2017), it might be the
case that recent discussions of best practices have improved lab
practices, but older reports could still have applied this seemingly
innocuous practice of adding participants to \enquote{bring out} the
effect of interest.

\section{Methods}\label{methods}

\subsection{Data}\label{data}

All data presented and analyzed in the present paper are part of a
standardized collection of meta-analyses (MetaLab), and are freely
available via the companion website \url{http://metalab.stanford.edu}.
Currently, MetaLab contains 13 meta-analyses, or datasets, where core
parts of each meta-analysis are standardized to allow for the
computation of common effect size estimates and for analyses that span
across different phenomena. These standardized variables include study
descriptors (such as citation and peer review status), participant
characteristics (including mean age, native language), methodological
information (for example what dependent variable was measured), and
information necessary to compute effect sizes (number of participants,
if available means and standard deviations of the dependent measure,
otherwise test statistics of the key hypothesis test, such as \emph{t}
values or \emph{F} scores). This way, the analyses presented in this
paper become possible.

MetaLab contains datasets that address phenomena ranging from
infant-directed speech preference to mutual exclusivity, sampled
opportunistically. Meta-analyses are either based on data made available
on MetaLab by their original authors (n=11 datasets) or they were
extracted from previously published meta-analyses related to language
development (n=2, Colonnesi, Stams, Koster, \& Noom, 2010; Dunst et al.,
2012). In the former case, the original authors attempted to document as
much detail as possible for each entered experiment (note that a paper
can contain many experiments, as shown in Table 1), as recommended for
reproducible and dynamic meta-analyses (Tsuji, Bergmann, \& Cristia,
2014). Detailed descriptions of all phenomena covered by MetaLab,
including which papers and other sources have been considered, can be
found at \url{http://metalab.stanford.edu}.

\subsection{Statistical approach}\label{statistical-approach}

As dependent measure, we report Cohen's \emph{d}, a standardized effect
size based on comparing sample means and their variance. Effect size was
calculated when possible from means and standard deviations across
designs with the appropriate formulae. When these data were not
available, we used test statistics, more precisely \emph{t} values or
\emph{F} scores of the test assessing the main hypothesis. We also
computed effect size variance, which allows to weight each effect size
when aggregating across studies. The variance is mainly determined by
the number of participants; intuitively effect sizes based on larger
samples will be assigned more weight. Note that for research designs
testing the same participants in two conditions (for example measuring
reactions of the same infants to infant- and adult-directed speech),
correlations between those two measures are needed to estimate the
effect size variance. This measure is usually not reported, despite
being necessary for effect size calculation. Some correlations could be
obtained through direct contact with the original authors (see e.g.,
Bergmann \& Cristia, 2016 for details), the remaining ones were imputed.
We report details of effect size calculation in the supplementary
materials and make available all scripts used in the present paper.

\subsubsection{Meta-analytic model}\label{meta-analytic-model}

Meta-analytic effect sizes were estimated using random-effect models
where effect sizes were weighted by their inverse variance. We further
used a multilevel approach, which takes into account not only the effect
sizes and variance of single studies, but also that effect sizes from
the same paper will be based on more similar studies than effect sizes
from different papers (Konstantopoulos, 2011). We relied on the
implementation in the metafor package (Viechtbauer, 2010) of R (R Core
Team, 2016). Excluded as outliers were effect sizes more than three
standard deviations away from the median effect size within each
dataset, thus accounting for the difference in median effect size across
phenomena.

\subsubsection{Power calculation}\label{power-calculation}

We calculated typical power using the pwr package (Champely, 2015) based
on the meta-analytical effect size and the median number of participants
within each phenomenon. This approach is insightful, because
meta-analytic effect size estimates are (typically) more reliable than
those of single studies. For targeted analyses of the power of the
seminal paper, we extracted the largest effect size and used this value
for power calculation, taking in both cases the median number of
participants in a meta-analysis into account.

\section{Results}\label{results}

\subsection{Statistical power}\label{statistical-power-1}

Table 1 provides a summary of typical sample sizes and effect sizes by
phenomenon. We remind the reader that recommendations are for this value
to be above 80\%, which refers to a likelihood that four out of five
studies show a significant outcome for an effect truly present in the
population.

As could be expected, sample sizes are small across all phenomena, with
the overall median in our data being 17. Effect sizes tend to fall into
ranges of small to medium effects, as defined by Cohen (Cohen, 1988).
The overall median effect size of all datasets is Cohen's \emph{d} =
0.69. As a result of those two factors, studies are typically severely
under-powered: Assuming a paired t-test (within-participant designs are
the most frequent in the present data) it is possible to detect an
effect in 80\% of all studies when Cohen's \emph{d} = 0.72; in other
words, this sample size would be appropriate when investigating a medium
to large effect. When comparing two independent groups, the effect size
that would be detectable with a sample size of 17 participants per group
increases to Cohen's \emph{d} = 0.99, a large effect that is rarely
observed as meta-analytic effect size in the present collection of
developmental meta-analyses.

Inversely, to detect the typical effect of Cohen's \emph{d} = 0.69,
studies would have to test 18 participants in a paired design; 1 more
than are included on average. It should be noted that this disparity
between observed and necessary sample size varies greatly across
phenomena, leading to drastic differences in observed power to detect
the main effect at stake. While studies on phonotactic learning and word
segmentation apparently typically run dramatically underpowered studies
(with typical power being under 10\%), experiments on gaze following and
online word recognition are very highly powered (95\% and 99\%,
respectively).

\begin{table}[tbp]
\begin{center}
\begin{threeparttable}
\caption{Descriptions of meta-analyses.}
\begin{tabular}{lrrrrrr}
\toprule
Topic & Age (in Months) & Median Sample Size (Range) & N Effect Sizes & N Papers & Cohen's d (SE) & Average Power\\
\midrule
Infant directed speech preference & 4.34 & 20 (10, 60) & 48 & 16 & 0.73 (0.13) & 0.61\\
Vowel discrimination (native) & 6.54 & 12 (6, 50) & 112 & 29 & 0.69 (0.09) & 0.37\\
Vowel discrimination (non-native) & 7.69 & 16 (8, 30) & 46 & 14 & 0.79 (0.24) & 0.58\\
Sound symbolism & 7.89 & 20 (11, 40) & 44 & 11 & 0.22 (0.11) & 0.10\\
Statistical sound category learning & 8.16 & 14.75 (5, 35) & 16 & 9 & -0.26 (0.16) & 0.10\\
Word segmentation & 8.29 & 20 (4, 64) & 284 & 68 & 0.16 (0.03) & 0.08\\
Phonotactic learning & 10.69 & 18 (8, 40) & 47 & 15 & 0.12 (0.07) & 0.06\\
Label advantage in concept learning & 12.36 & 13 (9, 32) & 48 & 15 & 0.45 (0.08) & 0.20\\
Gaze following & 14.24 & 23 (12, 63) & 32 & 11 & 1.08 (0.16) & 0.95\\
Online word recognition & 18.00 & 25 (16, 95) & 14 & 6 & 1.24 (0.26) & 0.99\\
Pointing and vocabulary (concurrent) & 22.22 & 24.5 (6, 50) & 12 & 12 & 0.98 (0.18) & 0.92\\
Mutual exclusivity & 23.99 & 16 (8, 72) & 58 & 19 & 0.81 (0.14) & 0.61\\
Categorization Bias & 42.00 & 14 (8, 20.5) & 77 & 9 & 0.27 (0.39) & 0.11\\
\bottomrule
\end{tabular}
\end{threeparttable}
\end{center}
\end{table}

\textbf{NOTE: Will ad summary row in the bottom by hand}

\subsubsection{The role of participant
age}\label{the-role-of-participant-age}

Participant age can be assumed to interact with effect size both for
conceptual and practical reasons. Younger participants might show
smaller effects in general because they are more immature in terms of
their information processing abilities, and they are not yet as
experienced with, and proficient in, their native language in
particular. As to practical reasons, measurements might be more noisy
for younger participants, as they could be a more difficult population
to recruit and test. We find no linear relationship between participant
age and sample size, effect size, and derived power on the level of
meta-analyses. In addition, the prediction that older participants might
be easier to recruit and test is not reflected in the observed sample
sizes. However, the only two datasets, gaze following and online word
recognition, with power over 80\% typically test participants older than
one year.

\subsubsection{Seminal papers as basis for sample size
planning}\label{seminal-papers-as-basis-for-sample-size-planning}

As Table 1 shows, experimenters are frequently not including a
sufficient number of participants to observe a given effect -- assuming
the meta-analytic estimate is accurate. It might, however, be possible,
that power has been determined based on a seminal paper to be replicated
and/or expanded. Initial reports tend to overestimate effect sizes
(Jennions \& Møller, 2002), possibly explaining the lack of power in
some datasets and studies.

We extracted for each dataset the oldest paper and therein the largest
reported effect size and re-calculated power accordingly, using the
median sample size of a given dataset. The results are shown in Table 2.
It turns out that in some cases, such as native and non-native vowel
discrimination, sample size choices match well with the oldest report.
The difference in power, noted in the last column, can be substantial,
with native vowel discrimination and phonotactic learning being the two
most salient examples. Here, sample sizes match well with the oldest
report and studies would be appropriately powered if this estimate were
representative of the true effect. For four datasets neither the seminal
paper nor meta-analytic effect size seem to be basis for sample size
decisions.

\begin{table}[tbp]
\begin{center}
\begin{threeparttable}
\caption{For each meta-analysis, largest effect size Cohen's *d* and derived power based on the first paper, along with the difference between power based on meta-analytic and oldest effect size.}
\begin{tabular}{lrrrrl}
\toprule
Meta-analysis (MA) & Oldest Effect Size & Meta-analytic Effect Size & Sample Size & Power based on 
 first report & Difference to
 meta-analytic power\\
\midrule
Statistical sound category learning & 0.56 & -0.26 & 15 & 0.31 & 0.211\\
Word segmentation & 0.56 & 0.16 & 20 & 0.40 & 0.326\\
Mutual exclusivity & 0.70 & 0.81 & 16 & 0.48 & -0.125\\
Label advantage in concept learning & 0.86 & 0.45 & 13 & 0.56 & 0.361\\
Pointing and vocabulary (concurrent) & 0.65 & 0.98 & 24 & 0.61 & -0.310\\
Vowel discrimination (non-native) & 1.02 & 0.79 & 16 & 0.80 & 0.220\\
Phonotactic learning & 0.98 & 0.12 & 18 & 0.81 & 0.748\\
Sound symbolism & 0.95 & 0.22 & 20 & 0.84 & 0.732\\
Online word recognition & 0.89 & 1.24 & 25 & 0.87 & -0.122\\
Gaze following & 1.29 & 1.08 & 23 & 0.99 & 0.043\\
Vowel discrimination (native) & 1.87 & 0.69 & 12 & 0.99 & 0.626\\
Infant directed speech preference & 2.39 & 0.73 & 20 & 1.00 & 0.388\\
Categorization Bias & 9.06 & 0.27 & 14 & 1.00 & 0.893\\
\bottomrule
\end{tabular}
\end{threeparttable}
\end{center}
\end{table}

\subsection{Method choice}\label{method-choice-1}

In most of our meta-analyses, multiple methods were used to tap into the
phenomenon at stake. Choosing a robust method can help increase power,
because more precise measurements lead to larger effects and thus
require fewer participants to be tested. However, the number of
participants relates to the final sample and not how many participants
had to be invited into the lab. We thus first quantify whether methods
differ in their typical drop-out rate, as economic considerations might
drive method choice. To this end we consider all methods across datasets
which have more than 10 associated effect sizes and for which
information on the number of dropouts was reported; this information is
not always reported in published papers. In the case of the two
meta-analyses we added based on published reports, the information of
drop-out rates was not available. Therefore, the following analyses only
cover 6 methods and 224 data points.

\subsubsection{Drop-out rates across
procedures}\label{drop-out-rates-across-procedures}

The results of a linear mixed effects model predicting dropout rate by
method and mean participant age (while controlling for the different
phenomena and associated underlying effect sizes being tested) are
summarized in the table below. The results show that, taking the most
frequently used method central fixation as the baseline, conditioned
headturn and stimulus alternation have significantly more drop-outs,
while forced choice has significantly fewer. Figure 1 underlines this
observation. Overall, stimulus alternation leads to the highest drop-out
rates, which lies at around 50\% (see Figure 1), and forced choice to
the lowest. Participant age interacts with the different methods. We
observe an increase in drop-out rates, which is most prominent in
conditioned headturn (a significant interaction) and headturn preference
procedure (where the interaction approaches significance).

Interestingly, the methods with lower drop-out rates, namely central
fixation and headturn preference procedure, are among the most frequent
ones in our data and certainly more frequent than those with higher
drop-out rates. The proportion of participants that can be retained
might thus indeed inform researchers' choice. This observation points to
the previously mentioned limitations regarding the participant pool, as
more participants will have to be tested to arrive at the same final
sample size.

\begin{table}[tbp]
\begin{center}
\begin{threeparttable}
\caption{Linear mixed effects model predicting dropout rate by method and participant age while accounting for the specific phenomenon.}
\begin{tabular}{llll}
\toprule
 & \multicolumn{1}{c}{Estimate} & \multicolumn{1}{c}{Std. Error} & \multicolumn{1}{c}{t value}\\
\midrule
(Intercept) & 32.8372364547405 & 5.18270575595169 & 6.33592528710143\\
methodconditioned head-turn & 41.7568152334524 & 9.70666233683515 & 4.30187162017497\\
methodforced-choice & -27.2548832121313 & 8.87560255001973 & -3.07076427302062\\
methodhead-turn preference procedure & 1.40027320022283 & 6.35607050157499 & 0.220304856573861\\
methodlooking while listening & -8.61485990923845 & 6.92980597857125 & -1.2431603331865\\
methodstimulus alternation & 20.3310364668288 & 6.33917912667968 & 3.20720333982386\\
ageC & 0.419580203901804 & 0.439671895960652 & 0.95430298765185\\
methodconditioned head-turn:ageC & 2.87744919382107 & 1.16473791428906 & 2.47046924335543\\
methodforced-choice:ageC & -0.2158939800302 & 0.647747533970996 & -0.333299578474146\\
methodhead-turn preference procedure:ageC & 0.963028502288058 & 0.719580640550116 & 1.33831908200286\\
methodlooking while listening:ageC & -0.567347683657796 & 0.798508880012725 & -0.710508922140922\\
methodstimulus alternation:ageC & -0.261530192918133 & 0.907336467333541 & -0.288239481530718\\
\bottomrule
\end{tabular}
\end{threeparttable}
\end{center}
\end{table}

\begin{figure}[htbp]
\centering
\includegraphics{ChildDevShort_files/figure-latex/DropoutPlot-1.pdf}
\caption{\label{fig:DropoutPlot}Percent dropout as explained by different
methods. CF = central fixation, CondHT = conditioned headturn, FC =
forced choice, HPP = headturn preference procedure, LwL = looking while
listening, SA = stimulus alternation.}
\end{figure}

Methods which retain a higher percentage of participants might either be
more suitable, because they are decreasing noise as most participants
are on task, or less selective, thus increasing noise as participants
who for example are fussy are more likely to enter the data pool. We
thus turn to a meta-analytic assessment of the same methods discussed
here.

\subsubsection{Effect sizes as a function of
procedure}\label{effect-sizes-as-a-function-of-procedure}

We built a meta-analytic model with Cohen's \emph{d} as the dependent
variable, method and mean age centered as independent variables, which
we allowed to interact. The model includes the variance of \emph{d} for
sampling variance, and paper within meta-analysis as a random effect
nested within phenomenon (because we assume that within a paper
experiments and thus effect sizes will be more similar to each other
than across papers). We again selected the most frequently used method
central fixation as the baseline and limited this analysis to the same
methods that we investigated above.

\begin{table}[tbp]
\begin{center}
\begin{threeparttable}
\caption{Meta-analytic regression predicting effect size Cohen's *d* with participant age and method (central fixation is baseline method).}
\begin{tabular}{lllllll}
\toprule
 & \multicolumn{1}{c}{estimate} & \multicolumn{1}{c}{se} & \multicolumn{1}{c}{zval} & \multicolumn{1}{c}{pval} & \multicolumn{1}{c}{ci.lb} & \multicolumn{1}{c}{ci.ub}\\
\midrule
intrcpt & 0.2240 & 0.1440 & 1.56 & 0.1198 & -0.0582 & 0.5063\\
ageC & 0.0112 & 0.0064 & 1.75 & 0.0793 & -0.0013 & 0.0237\\
relevel(method, "central fixation")conditioned head-turn & 1.8230 & 0.6047 & 3.01 & 0.0026 & 0.6379 & 3.0082\\
relevel(method, "central fixation")forced-choice & 0.5223 & 0.1868 & 2.80 & 0.0052 & 0.1561 & 0.8885\\
relevel(method, "central fixation")head-turn preference procedure & 0.1832 & 0.1163 & 1.58 & 0.1152 & -0.0447 & 0.4110\\
relevel(method, "central fixation")looking while listening & 0.4402 & 0.2427 & 1.81 & 0.0697 & -0.0355 & 0.9158\\
relevel(method, "central fixation")stimulus alternation & -0.0626 & 0.2745 & -0.23 & 0.8197 & -0.6007 & 0.4755\\
ageC:relevel(method, "central fixation")conditioned head-turn & 0.1144 & 0.0629 & 1.82 & 0.0690 & -0.0089 & 0.2376\\
ageC:relevel(method, "central fixation")forced-choice & -0.0089 & 0.0065 & -1.36 & 0.1725 & -0.0216 & 0.0039\\
ageC:relevel(method, "central fixation")head-turn preference procedure & 0.0091 & 0.0098 & 0.94 & 0.3497 & -0.0100 & 0.0283\\
ageC:relevel(method, "central fixation")looking while listening & 0.0250 & 0.0111 & 2.24 & 0.0248 & 0.0032 & 0.0468\\
ageC:relevel(method, "central fixation")stimulus alternation & 0.0039 & 0.0281 & 0.14 & 0.8895 & -0.0512 & 0.0590\\
\bottomrule
\end{tabular}
\end{threeparttable}
\end{center}
\end{table}

\begin{figure}[htbp]
\centering
\includegraphics{ChildDevShort_files/figure-latex/MethodPlot-1.pdf}
\caption{\label{fig:MethodPlot}Effect size by different methods. CF =
central fixation, CondHT = conditioned headturn, FC = forced choice, HPP
= headturn preference procedure, LwL = looking while listening, SA =
stimulus alternation.}
\end{figure}

The model results in Table 2 show that compared to central fixation,
conditioned headturn and forced choice yield reliably higher effect
sizes, all other methods do not statistically differ from this baseline
(note that looking while listening is approaching significance). When
factoring in age, looking while listening shows a significant
interaction, and conditioned headturn approaches significance,
indicating an increase in effect sizes as infants mature. Age is
marginally above the significance threshold, the positive estimate
further underlines that overall effect sizes increase for older
participants -- an observation consistent with the view that infants and
toddlers become more proficient language users and are increasingly able
to react appropriately in the lab.

\subsection{Questionable research
practices}\label{questionable-research-practices-1}

To assess whether researchers selectively add participants to obtain a
significant \emph{p} value, we assess the relationship between
(absolute) observed effect sizes in single studies and the associated
sample size. The rationale behind this analysis is simple: The smaller
the effect size, the larger the sample needed for a significant \emph{p}
value. If sample size decisions are made before data collection and all
results are published, we expect no relation between observed effect
size and sample size. A significant non-parametric correlation indicates
that only those studies with significant outcomes were published (Begg
\& Mazumdar, 1994).

\begin{figure}[htbp]
\centering
\includegraphics{ChildDevShort_files/figure-latex/Bias-1.pdf}
\caption{\label{fig:Bias}For each dataset observed effect size per study
plotted against sample size.}
\end{figure}

We illustrate the relationship between effect size and sample size,
separated by meta-analysis, in Figure XX. The regression line is plotted
on top of points indicating single experiments. Four datasets turn out
to have a significant negative relationship between sample size and
effect size, indicating bias; two assessing infants' ability to
discriminate vowels, one on word segmentation, and one testing whether
children use mutual exclusivity during word learning. The last case
might be driven by a single high-powered study, however. We further
observe a positive relationship between sample size and observed effect
size in two datasets, namely infant directed speech preference and
categorization bias.

\begin{table}[tbp]
\begin{center}
\begin{threeparttable}
\caption{Non-parametric correlations between sample sizes and effect sizes for each dataset. A significant value indicates bias.}
\begin{tabular}{lrr}
\toprule
Meta-analysis & Kendall's Tau & p-value\\
\midrule
Phonotactic learning & -0.207 & 0.052\\
Statistical sound category learning & 0.205 & 0.277\\
Categorization Bias & 0.151 & 0.07\\
Gaze following & 0.085 & 0.512\\
Infant directed speech preference & 0.010 & 0.921\\
Label advantage in concept learning & -0.057 & 0.59\\
Mutual exclusivity & -0.214 & 0.024\\
Vowel discrimination (native) & -0.283 & < .001\\
Vowel discrimination (non-native) & -0.229 & 0.032\\
Pointing and vocabulary (concurrent) & -0.154 & 0.491\\
Sound symbolism & -0.042 & 0.698\\
Online word recognition & -0.128 & 0.539\\
Word segmentation & -0.098 & 0.023\\
\bottomrule
\end{tabular}
\end{threeparttable}
\end{center}
\end{table}

\section{Discussion}\label{discussion}

In this paper, we made use of a collection of standardized meta-analyses
to assess the status quo in developmental research regarding typical
effect sizes, sample size, power, and methodological choices in 12
meta-analyses on language development. With an average meta-analytic
effect size of .57 and a typical sample size of only 17 participants per
cell, we find that power is at 60\%.

This means studies on language development, the sub-domain of
developmental research that the present collection of meta-analyses is
focused on, are severely under-powered. This is particularly salient for
phenomena typically tested on younger children, because sample sizes and
effect sizes are both small; the one exception for research topics
tested mainly with participants younger than one year is non-native
vowel discrimination, which can be attributed to a large meta-analytic
effect size estimate. Phenomena targeting older children tend towards
larger effects, and here some studies turn out to be high-powered (see
for example online word recognition). Both observations are first
indicators that effect size estimates might not be considered when
determining sample size. It might, in the case of apparently
over-powered studies however be possible that next to testing a main
effect, such as whether children recognize a given word online, studies
aimed to tap into factors affecting this ability. In this case, studies
would be powered appropriately, as an interaction effect will be more
difficult to detect than a main effect. Nonetheless, by and large, we
find that studies are habitually underpowered

We investigated the alternative possibility that researchers base their
sample size on the effect size reported in the seminal paper of their
research topic. This turns out to be an unsuitable strategy: As
described in the results section, the larger the original effect size,
the more likely is an overestimation of the meta-analytic effect size.
Researchers might thus be wary of reports implying a strong, robust
effect with infants and toddlers in the absence of corroborating data.
The lack of a relationship between either overall meta-analytic effect
size or seminal reported effect size and sample size across phenomena
indicates that researchers' experiment planning is not impacted by an
estimated effect size of the phenomenon under investigation. Studies
might instead be designed and conducted with pragmatic considerations in
mind, such as participant availability.

To help researchers choose the most efficient experiment design, and
thus potentially improve their power due to the use of a more sensitive
measure, we next turned to methods. Our investigation of method choice
considered both drop-out rates and whether effect sizes are differing
across methods. Overall, drop-out rates varied a great deal (with
medians between 5.9\% for forced-choice and 45\% for stimulus
alternation). However, high drop-out rates might be offset by high
effect sizes -- at least in the case of conditioned headturn. While
drop-out rates are around 30-50\%, effect sizes are above 1. Stimulus
alternation, in contrast, does not fall into this pattern of high
drop-out rates being correlated with high effect sizes, as the observed
effect sizes associated with this method are in the range typical for
meta-analyses in our dataset. The interpretation of this finding might
be that some methods, specifically conditioned headturn, which have
higher dropout rates, are better at generating high effect sizes due to
decreased noise (e.g., by excluding participants that are not on task).
However, there is an important caveat: Studies with fewer participants
(thanks to higher drop-out rates) might simply be underpowered, and thus
any significant finding is likely to over-estimate the effect. We
conclude thus that current efforts to estimate the impact of method
choice experimentally are an important endeavor in developmental
research (Frank et al., 2016).

A final set of analyses assessed the relationship between observed
effect size and sample size. This analysis might reflect whether
researchers selectively add participants to obtain a significant result.
We observed that in four datasets smaller effect sizes coincided with
larger sample sizes, which might be an indication of questionable
research practices. At the same time we find two (numerically) positive
correlations, an unexpected result as it means that larger sample sizes
coincide with larger effects. One possible reason for this might be that
for example older infants are both easier to test and yield larger
effects. This explanation is in line with our finding when investigating
the effect of method that higher participant age is linked to larger
effect sizes.

For the observed negative correlations alternative explanations to
questionable research practices are also possible: As soon as
researchers are aware that they are measuring a more subtle effect (for
example when selecting a contrast that is acoustically more difficult to
distinguish or when testing younger infants) and adjust sample sizes
accordingly, we expect to observe this negative correlation. In fact, in
the presence of consequent and accurate a priori power calculations, a
correlation between sample size and effect size must be observed.
However, our previous analyses indicate that power is not considered
when making sample size decisions.

We have assessed the same datasets for other indicators of questionable
research practices and publication bias, namely p-curves
({\textbf{???}}) and funnel plot asymmetry. For three datasets that
showed a negative correlation between sample size and effect size, we
also observe funnel plot asymmetry (both datasets on vowel
discrimination as well as mutual exclusivity). For those three datasets
we can thus conclude that publication bias underlies the observed link
between sample size and effect size.

\subsection{Concrete recommendations for developmental
scientists}\label{concrete-recommendations-for-developmental-scientists}

In this section, we aim to show how to move on from the status quo and
improve the reliability of developmental research.

\subsubsection{1. Calculate power
prospectively}\label{calculate-power-prospectively}

Our results indicate that most studies testing infants and toddlers are
severely underpowered when aiming to detect a main effect. Interactions
will show smaller effect sizes and thus will be even harder to detect in
most cases. Further, power varies greatly across phenomena, which mostly
is due to differences in effect sizes. Sample sizes are not adjusted
accordingly across phenomena, but remain close to the typical sample
size of 17.

Our first recommendation is thus to assess in advance how many
participants would be needed to detect an effect. Note that we based our
power estimations on whole meta-analyses, an analysis approach most
suitable to make general statements about the status quo. It might,
however, be the case that specific studies might want to base their
power estimates on a subset of effect sizes to match age group and
method. Both factors can, as we showed in our results, influence the to
be expected effect size. To facilitate such analyses, all meta-analyses
are shared on MetaLab and for each as much detail pertaining procedure
and measurements have been coded as possible (see also Tsuji et al.,
2014).

In lines of research where no meta-analytic effect size estimate is
available -- either because it is a novel phenomenon being investigated
or simply due to the absence of meta-analyses -- we recommend
considering typical effect sizes for the method used and the age group
being tested. This paper is a first step towards establishing such
measures, but more efforts and investigations are needed for robust
estimates (Cristia, Seidl, Singh, \& Houston, 2016; see for example
Frank et al., 2016; ManyBabies Collaborative, 2017).

\subsubsection{2. Carefully consider method
choice}\label{carefully-consider-method-choice}

One way to increase power is the use of more sensitive measurements; as
mentioned above, we do find striking differences between methods. When
possible, it can thus be helpful to consider the paradigm being used,
and possibly use a more sensitive way of measuring infants'
capabilities. One reason that researchers do not choose the most robust
methods might be due to a lack of consideration of meta-analytic effect
size estimates, which in turn might be (partially) due to a lack of
information on and experience in how to interpret effect size estimates
and use them for study planning (Mills-Smith, Spangler, Panneton, \&
Fritz, 2015).

\subsubsection{3. Report all data}\label{report-all-data}

A possible reason for prospective power calculations and meta-analyses
being rare lies in the availability of data in published reports.
Reports and discussions of effect sizes in experimental studies are
rare, but despite long-standing recommendations to move beyond the
persistent focus on \emph{p} values (such as American Psychological
Association, 2001), a shift towards effect sizes or even the reporting
of them has not (yet) been widely adopted (Mills-Smith et al., 2015).

A second impediment to meta-analyses in developmental science are
current reporting standards, which make it difficult and at times even
impossible to compute effect sizes from the published literature. For
example, for within-participant measures it is necessary to report the
correlation between conditions if two types of results are reported
(most commonly outcomes of a treatment and control condition). However,
this correlation, necessary to both compute effect sizes and their
variance, is habitually not reported and has to be obtained via direct
contact with the original authors (see for example Bergmann \& Cristia,
2016) or estimated (as described in Black \& Bergmann, 2017). In
addition, reporting (as well as analysis) of results is generally highly
variable, with raw means and standard deviations not being available for
all papers.

We suggest reporting the following information, in line with current APA
guidelines: Means and standard deviations of dependent measures being
statistically analyzed (for within-participant designs with two
dependent variables, correlations between the two should be added), test
statistic, exact \emph{p} value (when computed), and effect sizes (for
example Cohen's \emph{d} as used in the present paper) where possible.
Such a standard not only follows extant guidelines but also creates
coherence across papers and reports, thus improving clarity (Mills-Smith
et al., 2015). A step further would be the supplementary sharing of all
anonymized results on the participant level, thus allowing for the
necessary computations and opening the door for other types of
cumulative analyses, for example in direct replications comparing raw
results.

\subsection{How to increase the use and availability of
meta-analyses}\label{how-to-increase-the-use-and-availability-of-meta-analyses}

Conducting a meta-analysis is a laborious process, particularly
according to common practice where only a few people do the work, with
little support tools and educational materials available. Incentives for
creating meta-analyses are low, as public recognition is tied to a
single publication. The benefits of meta-analyses for the field, for
instance the possibility to conduct power analyses, are often neither
evident nor accessible to individual researchers, as the data are not
shared and traditional meta-analyses remain static after publication,
aging quickly as new results emerge (Tsuji et al., 2014).

To support the improvement current practices, we propose to make
meta-analyses available in the form of ready-to-use online tools,
dynamic reports, and as raw data. These different levels allow
researchers with varying interest and expertise interests to make the
best use of the extant record on language development, including study
planning by choosing robust methods and appropriate sample sizes. There
are additional advantages for interpreting single results as well as for
theory building that emerge from our collection of meta-analyses: On one
hand, researchers can easily check whether their study result falls
within the expected range of outcomes for their research question --
indicating whether or not a potential moderator influenced the result.
On the other hand, aggregating over many data points allows for the
tracing of emerging abilities over time, quantifying their growth, and
identifying possible trajectories and dependencies across phenomena (for
a demonstration see Lewis et al., 2016). Finally, by making our data and
source code open, we also invite contributions and can update our data,
be it by adding new results, file-drawer studies, or new datasets. Our
implementation of this proposal is freely online available at
\url{http://metalab.stanford.edu}.

\subsection{\texorpdfstring{Cumulative evidence to decide whether skills
are \enquote{absent} or
not}{Cumulative evidence to decide whether skills are absent or not}}\label{cumulative-evidence-to-decide-whether-skills-are-absent-or-not}

Developmental research often relies on interpreting both significant and
non-significant findings, particularly to establish a developmental
time-line tracing when skills emerge. This approach is problematic for
multiple reasons, as we mentioned in the introduction. Disentangling
whether a non-significant finding indicates the absence of a skill,
random measurement noise, or the lack of experimental power to detect
this skill reliably and with statistical support is in fact impossible
based on \emph{p} values. Further, we want to caution researchers
against interpreting the difference between significant and
non-significant findings without statistically assessing it first
(Gelman \& Stern, 2006).

Concretely, we recommend the use of meta-analytic tools as demonstrated
in this paper as well as in the work by Lewis et al. (2016). Aggregating
over multiple studies allows not only for a more reliable estimate of an
effect (because any single finding might either be a false positive or a
false negative) but also makes it possible to trace developmental
trajectories. A demonstration of such a procedure is given in the work
of Tsuji \& Cristia (2014) for native and non-native vowel
discrimination. Their results match well with the standard assumption
that infants begin to tune into their native language at around six
months of age. For a contrasting example, see Bergmann \& Cristia
(2016), where the typically assumed developmental trajectory for word
segmentation from native speech could not be confirmed, as across all
included age groups infants seem to be able to detect words in the
speech stream -- the effect size of this skill is simply comparatively
small and thus it is difficult to detect (see also Bergmann, Tsuji, \&
Cristia, 2017 for a more recent discussion of both meta-analyses).

\subsection{Future directions}\label{future-directions}

The present analyses can be expanded and improved in a number of ways.
First, the present collection of meta-analyses does not represent an
exhaustive survey of phenomena in language acquisition, let alone child
development research. Particularly, topics typically investigated in
younger children are over-represented. However, we sampled in an
opportunistic, and thus to some degree random fashion, which lends some
credibility to our approach. It would nonetheless be advisable to follow
up on this report with a larger sample. To this end, we made all source
materials along with extensive documentation available online.

Second, it would be important to further investigate the role of
participant age in child development research. It is possible that
developmental psychologists working with older age groups might focus on
different issues or find that power and experimental design choices are
less problematic; for instance, it may be easier to recruit larger
samples via institutional testing in schools, and older children may be
more reliable and consistent in their responses (Roberts \& DelVecchio,
2000). We thus hope particularly to analyze more studies of older
children to test this assumption.

\subsection{Conclusion}\label{conclusion}

We have demonstrated the use of standardized collections of
meta-analyses for a diagnosis of (potential) issues in developmental
research. Our results point to an overall lack of consideration of
meta-analytic effect size in experiment planning, leading to habitually
under-powered studies. In addition, method choice and participant age
play an important role in the to be expected outcome; we here provide
first estimates of the importance of either factor in experiment design.
Assessing data quality, we find no evidence for questionable research
practices and conclude that most phenomena considered here have
evidential value. To ensure that developmental research is robust and
that theories of child development are built on solid and reliable
results, we strongly recommend an increased use of effect sizes and
meta-analytic tools, including prospective power calculations.

\newpage

\section{References}\label{references}

\setlength{\parindent}{-0.5in} \setlength{\leftskip}{0.5in}

\hypertarget{refs}{}
\hypertarget{ref-APA2001}{}
American Psychological Association. (2001). \emph{Publication manual of
the american psychological association} (5th ed.). Washington, DC:
American Psychological Association.

\hypertarget{ref-pubbias}{}
Begg, C. B., \& Mazumdar, M. (1994). Operating characteristics of a rank
correlation test for publication bias. \emph{Biometrics}, 1088--1101.

\hypertarget{ref-InWordDB}{}
Bergmann, C., \& Cristia, A. (2016). Development of infants'
segmentation of words from native speech: A meta-analytic approach.
\emph{Developmental Science}, \emph{19}(6), 901--917.

\hypertarget{ref-TopDownBottomUp}{}
Bergmann, C., Tsuji, S., \& Cristia, A. (2017). Top-down versus
bottom-up theories of phonological acquisition: A big data approach.
\emph{Submitted}.

\hypertarget{ref-InStatDB}{}
Black, A., \& Bergmann, C. (2017). Quantifying infants' statistical word
segmentation: A meta-analysis. In \emph{Proceedings of the 39th annual
conference of the cognitive science society}. Cognitive Science Society.

\hypertarget{ref-button2013power}{}
Button, K. S., Ioannidis, J. P., Mokrysz, C., Nosek, B. A., Flint, J.,
Robinson, E. S., \& Munafò, M. R. (2013). Power failure: Why small
sample size undermines the reliability of neuroscience. \emph{Nature
Reviews Neuroscience}, \emph{14}(5), 365--376.

\hypertarget{ref-pwr}{}
Champely, S. (2015). \emph{pwr: Basic Functions for Power Analysis}.
Retrieved from \url{https://CRAN.R-project.org/package=pwr}

\hypertarget{ref-cohen}{}
Cohen, J. (1988). \emph{Statistical power analysis for the behavioural
sciences}. NJ: Lawrence Earlbaum Associates.

\hypertarget{ref-PointingMA}{}
Colonnesi, C., Stams, G. J. J., Koster, I., \& Noom, M. J. (2010). The
relation between pointing and language development: A meta-analysis.
\emph{Developmental Review}, \emph{30}(4), 352--366.

\hypertarget{ref-TestRetest}{}
Cristia, A., Seidl, A., Singh, L., \& Houston, D. (2016). Test--Retest
reliability in infant speech perception tasks. \emph{Infancy},
\emph{21}, 648--667.

\hypertarget{ref-IDS_MA}{}
Dunst, C., Gorman, E., \& Hamby, D. (2012). Preference for
infant-directed speech in preverbal young children. \emph{Center for
Early Literacy Learning}, \emph{5}(1), 1--13.

\hypertarget{ref-eason2017survey}{}
Eason, A. E., Hamlin, J. K., \& Sommerville, J. A. (2017). A survey of
common practices in infancy research: Description of policies,
consistency across and within labs, and suggestions for improvements.
\emph{Infancy}.

\hypertarget{ref-Manybabies}{}
Frank, M. C., Bergelson, E., Bergmann, C., Cristia, A., Floccia, C.,
Gervain, J., \ldots{} Yurovsky, D. (2016). A collaborative approach to
infant research: Promoting reproducibility, best practices, and
theory-building. \emph{Infancy}.

\hypertarget{ref-tablet}{}
Frank, M. C., Sugarman, E., Horowitz, A. C., Lewis, M. L., \& Yurovsky,
D. (2016). Using tablets to collect data from young children.
\emph{Journal of Cognition and Development}, \emph{17}(1), 1--17.

\hypertarget{ref-gelman2006difference}{}
Gelman, A., \& Stern, H. (2006). The difference between ``significant''
and ``not significant'' is not itself statistically significant.
\emph{The American Statistician}, \emph{60}(4), 328--331.

\hypertarget{ref-Ioannidis2005}{}
Ioannidis, J. P. (2005). Why most published research findings are false.
\emph{PLoS Med}, \emph{2}(8), e124.

\hypertarget{ref-jennions2002relationships}{}
Jennions, M. D., \& Møller, A. P. (2002). Relationships fade with time:
A meta-analysis of temporal trends in publication in ecology and
evolution. \emph{Proceedings of the Royal Society of London B:
Biological Sciences}, \emph{269}(1486), 43--48.

\hypertarget{ref-konstantopoulos2011}{}
Konstantopoulos, S. (2011). Fixed effects and variance components
estimation in three-level meta-analysis. \emph{Research Synthesis
Methods}, \emph{2}(1), 61--76.

\hypertarget{ref-SynthesisPaper}{}
Lewis, M. L., Braginsky, M., Tsuji, S., Bergmann, C., Piccinini, P. E.,
Cristia, A., \& Frank, M. C. (2016). A Quantitative Synthesis of Early
Language Acquisition Using Meta-Analysis. \emph{Preprint}. Retrieved
from \url{https://osf.io/htsjm/}

\hypertarget{ref-Manybabies1}{}
ManyBabies Collaborative. (2017). Quantifying sources of variability in
infancy research using the infant-directed speech preference.
\emph{Advances in Methods and Practices in Psychological Science}.

\hypertarget{ref-Mills-Smith2015}{}
Mills-Smith, L., Spangler, D. P., Panneton, R., \& Fritz, M. S. (2015).
A missed opportunity for clarity: Problems in the reporting of effect
size estimates in infant developmental science. \emph{Infancy},
\emph{20}(4), 416--432.

\hypertarget{ref-nosek2012scientific2}{}
Nosek, B. A., Spies, J. R., \& Motyl, M. (2012). Scientific utopia: II.
restructuring incentives and practices to promote truth over
publishability. \emph{Perspectives on Psychological Science},
\emph{7}(6), 615--631.

\hypertarget{ref-R}{}
R Core Team. (2016). \emph{R: A language and environment for statistical
computing}. Vienna, Austria: R Foundation for Statistical Computing.
Retrieved from \url{https://www.R-project.org/}

\hypertarget{ref-roberts2000rank}{}
Roberts, B. W., \& DelVecchio, W. F. (2000). The rank-order consistency
of personality traits from childhood to old age: A quantitative review
of longitudinal studies. \emph{Psychological Bulletin}, \emph{126}(1),
3.

\hypertarget{ref-Simmons2011}{}
Simmons, J. P., Nelson, L. D., \& Simonsohn, U. (2011). False-positive
psychology: Undisclosed flexibility in data collection and analysis
allows presenting anything as significant. \emph{Psychological Science},
\emph{22}(11), 1359--1366.

\hypertarget{ref-InPhonDB}{}
Tsuji, S., \& Cristia, A. (2014). Perceptual attunement in vowels: A
meta-analysis. \emph{Developmental Psychobiology}, \emph{56}(2),
179--191.

\hypertarget{ref-Tsuji2014}{}
Tsuji, S., Bergmann, C., \& Cristia, A. (2014). Community-augmented
meta-analyses: Toward cumulative data assessment. \emph{Psychological
Science}, \emph{9}(6), 661--665.

\hypertarget{ref-metafor}{}
Viechtbauer, W. (2010). Conducting meta-analyses in R with the metafor
package. \emph{Journal of Statistical Software}, \emph{36}(3), 1--48.
Retrieved from \url{http://www.jstatsoft.org/v36/i03/}






\end{document}
